\documentclass[]{article}
\usepackage{lmodern}
\usepackage{amssymb,amsmath}
\usepackage{ifxetex,ifluatex}
\usepackage{fixltx2e} % provides \textsubscript
\ifnum 0\ifxetex 1\fi\ifluatex 1\fi=0 % if pdftex
  \usepackage[T1]{fontenc}
  \usepackage[utf8]{inputenc}
\else % if luatex or xelatex
  \ifxetex
    \usepackage{mathspec}
  \else
    \usepackage{fontspec}
  \fi
  \defaultfontfeatures{Ligatures=TeX,Scale=MatchLowercase}
\fi
% use upquote if available, for straight quotes in verbatim environments
\IfFileExists{upquote.sty}{\usepackage{upquote}}{}
% use microtype if available
\IfFileExists{microtype.sty}{%
\usepackage{microtype}
\UseMicrotypeSet[protrusion]{basicmath} % disable protrusion for tt fonts
}{}
\usepackage[margin=1in]{geometry}
\usepackage{hyperref}
\hypersetup{unicode=true,
            pdftitle={Adding Custom Patient-Level Prediction Algorithms},
            pdfauthor={Jenna Reps, Martijn J. Schuemie, Patrick B. Ryan, Peter R. Rijnbeek},
            pdfborder={0 0 0},
            breaklinks=true}
\urlstyle{same}  % don't use monospace font for urls
\usepackage{color}
\usepackage{fancyvrb}
\newcommand{\VerbBar}{|}
\newcommand{\VERB}{\Verb[commandchars=\\\{\}]}
\DefineVerbatimEnvironment{Highlighting}{Verbatim}{commandchars=\\\{\}}
% Add ',fontsize=\small' for more characters per line
\usepackage{framed}
\definecolor{shadecolor}{RGB}{248,248,248}
\newenvironment{Shaded}{\begin{snugshade}}{\end{snugshade}}
\newcommand{\KeywordTok}[1]{\textcolor[rgb]{0.13,0.29,0.53}{\textbf{#1}}}
\newcommand{\DataTypeTok}[1]{\textcolor[rgb]{0.13,0.29,0.53}{#1}}
\newcommand{\DecValTok}[1]{\textcolor[rgb]{0.00,0.00,0.81}{#1}}
\newcommand{\BaseNTok}[1]{\textcolor[rgb]{0.00,0.00,0.81}{#1}}
\newcommand{\FloatTok}[1]{\textcolor[rgb]{0.00,0.00,0.81}{#1}}
\newcommand{\ConstantTok}[1]{\textcolor[rgb]{0.00,0.00,0.00}{#1}}
\newcommand{\CharTok}[1]{\textcolor[rgb]{0.31,0.60,0.02}{#1}}
\newcommand{\SpecialCharTok}[1]{\textcolor[rgb]{0.00,0.00,0.00}{#1}}
\newcommand{\StringTok}[1]{\textcolor[rgb]{0.31,0.60,0.02}{#1}}
\newcommand{\VerbatimStringTok}[1]{\textcolor[rgb]{0.31,0.60,0.02}{#1}}
\newcommand{\SpecialStringTok}[1]{\textcolor[rgb]{0.31,0.60,0.02}{#1}}
\newcommand{\ImportTok}[1]{#1}
\newcommand{\CommentTok}[1]{\textcolor[rgb]{0.56,0.35,0.01}{\textit{#1}}}
\newcommand{\DocumentationTok}[1]{\textcolor[rgb]{0.56,0.35,0.01}{\textbf{\textit{#1}}}}
\newcommand{\AnnotationTok}[1]{\textcolor[rgb]{0.56,0.35,0.01}{\textbf{\textit{#1}}}}
\newcommand{\CommentVarTok}[1]{\textcolor[rgb]{0.56,0.35,0.01}{\textbf{\textit{#1}}}}
\newcommand{\OtherTok}[1]{\textcolor[rgb]{0.56,0.35,0.01}{#1}}
\newcommand{\FunctionTok}[1]{\textcolor[rgb]{0.00,0.00,0.00}{#1}}
\newcommand{\VariableTok}[1]{\textcolor[rgb]{0.00,0.00,0.00}{#1}}
\newcommand{\ControlFlowTok}[1]{\textcolor[rgb]{0.13,0.29,0.53}{\textbf{#1}}}
\newcommand{\OperatorTok}[1]{\textcolor[rgb]{0.81,0.36,0.00}{\textbf{#1}}}
\newcommand{\BuiltInTok}[1]{#1}
\newcommand{\ExtensionTok}[1]{#1}
\newcommand{\PreprocessorTok}[1]{\textcolor[rgb]{0.56,0.35,0.01}{\textit{#1}}}
\newcommand{\AttributeTok}[1]{\textcolor[rgb]{0.77,0.63,0.00}{#1}}
\newcommand{\RegionMarkerTok}[1]{#1}
\newcommand{\InformationTok}[1]{\textcolor[rgb]{0.56,0.35,0.01}{\textbf{\textit{#1}}}}
\newcommand{\WarningTok}[1]{\textcolor[rgb]{0.56,0.35,0.01}{\textbf{\textit{#1}}}}
\newcommand{\AlertTok}[1]{\textcolor[rgb]{0.94,0.16,0.16}{#1}}
\newcommand{\ErrorTok}[1]{\textcolor[rgb]{0.64,0.00,0.00}{\textbf{#1}}}
\newcommand{\NormalTok}[1]{#1}
\usepackage{graphicx,grffile}
\makeatletter
\def\maxwidth{\ifdim\Gin@nat@width>\linewidth\linewidth\else\Gin@nat@width\fi}
\def\maxheight{\ifdim\Gin@nat@height>\textheight\textheight\else\Gin@nat@height\fi}
\makeatother
% Scale images if necessary, so that they will not overflow the page
% margins by default, and it is still possible to overwrite the defaults
% using explicit options in \includegraphics[width, height, ...]{}
\setkeys{Gin}{width=\maxwidth,height=\maxheight,keepaspectratio}
\IfFileExists{parskip.sty}{%
\usepackage{parskip}
}{% else
\setlength{\parindent}{0pt}
\setlength{\parskip}{6pt plus 2pt minus 1pt}
}
\setlength{\emergencystretch}{3em}  % prevent overfull lines
\providecommand{\tightlist}{%
  \setlength{\itemsep}{0pt}\setlength{\parskip}{0pt}}
\setcounter{secnumdepth}{5}
% Redefines (sub)paragraphs to behave more like sections
\ifx\paragraph\undefined\else
\let\oldparagraph\paragraph
\renewcommand{\paragraph}[1]{\oldparagraph{#1}\mbox{}}
\fi
\ifx\subparagraph\undefined\else
\let\oldsubparagraph\subparagraph
\renewcommand{\subparagraph}[1]{\oldsubparagraph{#1}\mbox{}}
\fi

%%% Use protect on footnotes to avoid problems with footnotes in titles
\let\rmarkdownfootnote\footnote%
\def\footnote{\protect\rmarkdownfootnote}

%%% Change title format to be more compact
\usepackage{titling}

% Create subtitle command for use in maketitle
\newcommand{\subtitle}[1]{
  \posttitle{
    \begin{center}\large#1\end{center}
    }
}

\setlength{\droptitle}{-2em}

  \title{Adding Custom Patient-Level Prediction Algorithms}
    \pretitle{\vspace{\droptitle}\centering\huge}
  \posttitle{\par}
    \author{Jenna Reps, Martijn J. Schuemie, Patrick B. Ryan, Peter R. Rijnbeek}
    \preauthor{\centering\large\emph}
  \postauthor{\par}
      \predate{\centering\large\emph}
  \postdate{\par}
    \date{2018-09-09}

\usepackage{fancyhdr}
\pagestyle{fancy}
\fancyhead{}
\fancyhead[CO,CE]{Custom Patient-Level Prediction Algorithms}
\fancyfoot[CO,CE]{PatientLevelPrediction Package Version 2.0.5}
\fancyfoot[LE,RO]{\thepage}
\renewcommand{\headrulewidth}{0.4pt}
\renewcommand{\footrulewidth}{0.4pt}

\begin{document}
\maketitle

{
\setcounter{tocdepth}{2}
\tableofcontents
}
\section{Introduction}\label{introduction}

This vignette describes how you can add your own custom algorithms in
the Observational Health Data Sciencs and Informatics (OHDSI)
\href{http://github.com/OHDSI/PatientLevelPrediction}{\texttt{PatientLevelPrediction}}
package. This allows you to fully leverage the OHDSI
PatientLevelPrediction framework for model development and validation.
This vignette assumes you have read and are comfortable with building
single patient level prediction models as described in the
\href{https://github.com/OHDSI/PatientLevelPrediction/blob/master/inst/doc/BuildingPredictiveModels.pdf}{\texttt{BuildingPredictiveModels}
vignette}.

\textbf{We invite you to share your new algorithms with the OHDSI
community through our
\href{http://github.com/OHDSI/PatientLevelPrediction}{GitHub
repository}.}

\section{Algorithm Code Structure}\label{algorithm-code-structure}

Each algorithm in the package should be implemented in its own
\textless{}Name\textgreater{}.R file, e.g.~KNN.R, containing a
set\textless{}Name\textgreater{} function and a
fit\textless{}Name\textgreater{} function. Furthermore, a corresponding
predict function in predict.R is needed (if there isn't one available
that would work, see example at the end of the document). We will now
describe each of these functions in more detail below.

\subsection{Set}\label{set}

The set\textless{}Name\textgreater{} is a function that takes as input
the different hyper-parameter values to do a grid search when training.
The output of the functions needs to be a list as class
\texttt{modelSettings} containing:

\begin{itemize}
\tightlist
\item
  param - all the combinations of the hyper-parameter values input
\item
  model - a string specifying what function to call to fit the model
\item
  name - a string containing the name of the model.
\end{itemize}

For example, if you were adding a model called madeUp that has two
hyper-parameters then the set function should be:

\begin{Shaded}
\begin{Highlighting}[]
\NormalTok{setMadeUp <-}\StringTok{ }\ControlFlowTok{function}\NormalTok{(}\DataTypeTok{a=}\DecValTok{1}\NormalTok{, }\DataTypeTok{b=}\DecValTok{2}\NormalTok{, }\DataTypeTok{seed=}\OtherTok{NULL}\NormalTok{)\{}
  \CommentTok{# add input checks here...}
  
  \CommentTok{# now create list of all combinations:}
\NormalTok{  result <-}\StringTok{ }\KeywordTok{list}\NormalTok{(}\DataTypeTok{model=}\StringTok{'fitMadeUp'}\NormalTok{, }\CommentTok{# this will be called to train the made up model}
                 \DataTypeTok{param=} \KeywordTok{split}\NormalTok{(}\KeywordTok{expand.grid}\NormalTok{(}\DataTypeTok{a=}\NormalTok{a, }
                                          \DataTypeTok{b=}\NormalTok{b,}
                                          \DataTypeTok{seed=}\KeywordTok{ifelse}\NormalTok{(}\KeywordTok{is.null}\NormalTok{(seed),}\StringTok{'NULL'}\NormalTok{, seed)),}
                              \DecValTok{1}\OperatorTok{:}\NormalTok{(}\KeywordTok{length}\NormalTok{(a)}\OperatorTok{*}\KeywordTok{length}\NormalTok{(b)  )),}
                 \DataTypeTok{name=}\StringTok{'Made Up Algorithm'}
\NormalTok{  )}
  \KeywordTok{class}\NormalTok{(result) <-}\StringTok{ 'modelSettings'} 
  
  \KeywordTok{return}\NormalTok{(result)}
\NormalTok{\}}
\end{Highlighting}
\end{Shaded}

\subsection{Fit}\label{fit}

This function should train your custom model for each parameter entry,
pick the best parameters and train a final model for that setting.

The fit\textless{}Model\textgreater{} should have as inputs:

\begin{itemize}
\tightlist
\item
  population - the study popualation the model is being developed on
\item
  plpData - the plpData object
\item
  param - the hyper-parameters as a list of all combinations
\item
  quiet - T or F indicating whether to output progress
\item
  outcomeId - the outcome id
\item
  cohortId - the target population id
\end{itemize}

The fit function should return a list of class \texttt{plpModel} with
the following objects:

\begin{itemize}
\tightlist
\item
  model - a trained model
\item
  modelSettings - a list containing the model and input param
\item
  trainCVAac - a value with the train AUC value
\item
  hyperParamSearch - a dataframe with the hyperparameter grid and
  corresponding AUCs
\item
  metaData - the metaData from the plpData object
\item
  populationSettings - the settings used to create the population and
  define the time-at-risk
\item
  outcomeId - the outcomeId being predicted
\item
  cohortId - the cohortId corresponding to the target cohort
\item
  varImp - a dataframe with the covaraites and a measure of importance
\item
  trainingTime - how long it took to develop/evaluate the model
\item
  covariateMap - if the plpData are converted to a matrix for model
  compatibility this tells us what covariate each row in the matrix
  correpsonds to and is need when implementing the model on new data
\end{itemize}

The plpModle returned by fit also has a type attribute, this points to
the predict function, for example
\texttt{attr(result,\ \textquotesingle{}type\textquotesingle{})\ \textless{}-\ \textquotesingle{}madeup\textquotesingle{}}
means when the model is applied to new data, the `predict.madeup'
function in Predict.R is called. if this doesnt exist, then the model
will fail. Another attribute is the predictionType
\texttt{attr(result,\ \textquotesingle{}predictionType\textquotesingle{})\ \textless{}-\ \textquotesingle{}binary\textquotesingle{}}
this is currently not needed but may be important in the future when we
expand to regression or multiclass classification.

For example:

\begin{Shaded}
\begin{Highlighting}[]
\NormalTok{fitMadeUp <-}\StringTok{ }\ControlFlowTok{function}\NormalTok{(population, plpData, param, }\DataTypeTok{quiet=}\NormalTok{F,}
\NormalTok{                        outcomeId, cohortId, ...)\{}
  
  \CommentTok{# **************** code to train the model here}
  \CommentTok{# trainedModel <- this code should apply each hyper-parameter using the cross validation}
  \CommentTok{#                 then pick out the best hyper-parameter setting}
  \CommentTok{#                 and finally fit a model on the whole train data using the }
  \CommentTok{#                 optimal hyper-parameter settings}
  \CommentTok{# ****************}
  
  \CommentTok{# construct the standard output for a model:}
\NormalTok{  result <-}\StringTok{ }\KeywordTok{list}\NormalTok{(}\DataTypeTok{model =}\NormalTok{ trainedModel,}
                 \DataTypeTok{modelSettings =} \KeywordTok{list}\NormalTok{(}\DataTypeTok{model=}\StringTok{'made_up'}\NormalTok{, }\DataTypeTok{modelParameters=}\NormalTok{param),}
                 \DataTypeTok{trainCVAuc =} \OtherTok{NULL}\NormalTok{,}
                 \DataTypeTok{hyperParamSearch =}\NormalTok{ hyperSummary,}
                 \DataTypeTok{metaData =}\NormalTok{ plpData}\OperatorTok{$}\NormalTok{metaData,}
                 \DataTypeTok{populationSettings =} \KeywordTok{attr}\NormalTok{(population, }\StringTok{'metaData'}\NormalTok{),}
                 \DataTypeTok{outcomeId=}\NormalTok{outcomeId,}\CommentTok{# can use populationSettings$outcomeId?}
                 \DataTypeTok{cohortId=}\NormalTok{cohortId,}
                 \DataTypeTok{varImp =} \OtherTok{NULL}\NormalTok{,}
                 \DataTypeTok{trainingTime=}\NormalTok{comp,}
                 \DataTypeTok{covariateMap=}\NormalTok{result}\OperatorTok{$}\NormalTok{map}
\NormalTok{  )}
  \KeywordTok{class}\NormalTok{(result) <-}\StringTok{ 'plpModel'}
  \KeywordTok{attr}\NormalTok{(result, }\StringTok{'type'}\NormalTok{) <-}\StringTok{ 'madeup'}
  \KeywordTok{attr}\NormalTok{(result, }\StringTok{'predictionType'}\NormalTok{) <-}\StringTok{ 'binary'}
  \KeywordTok{return}\NormalTok{(result)}
    
\NormalTok{\}}
\end{Highlighting}
\end{Shaded}

You could make the fitMadeUp function cleaner by adding helper function
in the MadeUp.R file that are called by the fit function. As the end of
the fit function specified
\texttt{attr(result,\ \textquotesingle{}type\textquotesingle{})\ \textless{}-\ \textquotesingle{}madeup\textquotesingle{}}
we also need to make sure there is a \texttt{predict.madeup} function in
Predict.R:

\subsection{Predict}\label{predict}

The prediction function takes as input the plpModel returned by fit, a
population and corresponding plpData. It returns a data.frame with the
columns:

\begin{itemize}
\tightlist
\item
  rowId - the id for each person in the population
\item
  value - the predicted risk from the plpModel
\end{itemize}

If the population contains the columns outcomeCount and indexes, then
these are also in the output.

For example:

\begin{Shaded}
\begin{Highlighting}[]
\NormalTok{predict.madeup <-}\StringTok{ }\ControlFlowTok{function}\NormalTok{(plpModel,population, plpData, ...)\{ }

  \CommentTok{# ************* code to do prediction for each rowId in population}
  \CommentTok{# prediction <- code to do prediction here returning columns: rowId }
  \CommentTok{#               and value (predicted risk)}
  \CommentTok{#**************}
  
\NormalTok{  prediction <-}\StringTok{ }\KeywordTok{merge}\NormalTok{(population, prediction, }\DataTypeTok{by=}\StringTok{'rowId'}\NormalTok{)}
\NormalTok{  prediction <-}\StringTok{ }\NormalTok{prediction[,}\KeywordTok{colnames}\NormalTok{(prediction)}\OperatorTok\KeywordTok{c}\NormalTok{(}\StringTok{'rowId'}\NormalTok{,}\StringTok{'outcomeCount'}\NormalTok{,}
                                                      \StringTok{'indexes'}\NormalTok{, }\StringTok{'value'}\NormalTok{)] }
  \KeywordTok{attr}\NormalTok{(prediction, }\StringTok{"metaData"}\NormalTok{) <-}\StringTok{ }\KeywordTok{list}\NormalTok{(}\DataTypeTok{predictionType =} \StringTok{"binary"}\NormalTok{) }
  \KeywordTok{return}\NormalTok{(prediction)}
  
\NormalTok{\}}
\end{Highlighting}
\end{Shaded}

\section{Algorithm Example}\label{algorithm-example}

Below a fully functional algorithm example is given, however we highly
recommend you to have a look at the available algorithms in the package.

\subsection{Set}\label{set-1}

\begin{Shaded}
\begin{Highlighting}[]
\NormalTok{setMadeUp <-}\StringTok{ }\ControlFlowTok{function}\NormalTok{(}\DataTypeTok{a=}\DecValTok{1}\NormalTok{, }\DataTypeTok{b=}\DecValTok{2}\NormalTok{, }\DataTypeTok{seed=}\OtherTok{NULL}\NormalTok{)\{}
  \CommentTok{# check a is valid positive value}
  \ControlFlowTok{if}\NormalTok{(}\KeywordTok{missing}\NormalTok{(a))\{}
    \KeywordTok{stop}\NormalTok{(}\StringTok{'a must be input'}\NormalTok{)}
\NormalTok{  \}}
  \ControlFlowTok{if}\NormalTok{(}\OperatorTok{!}\KeywordTok{class}\NormalTok{(a)}\OperatorTok\KeywordTok{c}\NormalTok{(}\StringTok{'numeric'}\NormalTok{,}\StringTok{'integer'}\NormalTok{)\{}
    \KeywordTok{stop}\NormalTok{(}\StringTok{'a must be numeric'}\NormalTok{)}
\NormalTok{  \}}
  \ControlFlowTok{if}\NormalTok{(a }\OperatorTok{<}\StringTok{ }\DecValTok{0}\NormalTok{)\{}
    \KeywordTok{stop}\NormalTok{(}\StringTok{'a must be positive'}\NormalTok{)}
\NormalTok{  \}}
  \CommentTok{# check b is numeric}
  \ControlFlowTok{if}\NormalTok{(}\KeywordTok{missing}\NormalTok{(b))\{}
    \KeywordTok{stop}\NormalTok{(}\StringTok{'b must be input'}\NormalTok{)}
\NormalTok{  \}}
  \ControlFlowTok{if}\NormalTok{(}\OperatorTok{!}\KeywordTok{class}\NormalTok{(b)}\OperatorTok\KeywordTok{c}\NormalTok{(}\StringTok{'numeric'}\NormalTok{,}\StringTok{'integer'}\NormalTok{)\{}
    \KeywordTok{stop}\NormalTok{(}\StringTok{'b must be numeric'}\NormalTok{)}
\NormalTok{  \}}
  
  \CommentTok{# now create list of all combinations:}
\NormalTok{  result <-}\StringTok{ }\KeywordTok{list}\NormalTok{(}\DataTypeTok{model=}\StringTok{'fitMadeUp'}\NormalTok{, }
                 \DataTypeTok{param=} \KeywordTok{split}\NormalTok{(}\KeywordTok{expand.grid}\NormalTok{(}\DataTypeTok{a=}\NormalTok{a, }
                                          \DataTypeTok{b=}\NormalTok{b,}
                                          \DataTypeTok{seed=}\KeywordTok{ifelse}\NormalTok{(}\KeywordTok{is.null}\NormalTok{(seed),}\StringTok{'NULL'}\NormalTok{, seed)),}
                              \DecValTok{1}\OperatorTok{:}\NormalTok{(}\KeywordTok{length}\NormalTok{(a)}\OperatorTok{*}\KeywordTok{length}\NormalTok{(b)  )),}
                 \DataTypeTok{name=}\StringTok{'Made Up Algorithm'}
\NormalTok{  )}
  \KeywordTok{class}\NormalTok{(result) <-}\StringTok{ 'modelSettings'} 
  
  \KeywordTok{return}\NormalTok{(result)}
    
  
\ErrorTok{\}}
\end{Highlighting}
\end{Shaded}

\subsection{Fit}\label{fit-1}

\begin{Shaded}
\begin{Highlighting}[]
\NormalTok{fitMadeUp <-}\StringTok{ }\ControlFlowTok{function}\NormalTok{(population, plpData, param, }\DataTypeTok{quiet=}\NormalTok{F,}
\NormalTok{                        outcomeId, cohortId, ...)\{}
    \ControlFlowTok{if}\NormalTok{(}\OperatorTok{!}\NormalTok{quiet)}
    \KeywordTok{writeLines}\NormalTok{(}\StringTok{'Training Made Up model'}\NormalTok{)}
  
  \ControlFlowTok{if}\NormalTok{(param[[}\DecValTok{1}\NormalTok{]]}\OperatorTok{$}\NormalTok{seed}\OperatorTok{!=}\StringTok{'NULL'}\NormalTok{)}
    \KeywordTok{set.seed}\NormalTok{(param[[}\DecValTok{1}\NormalTok{]]}\OperatorTok{$}\NormalTok{seed)}
  
    \CommentTok{# check plpData is coo format:}
  \ControlFlowTok{if}\NormalTok{(}\OperatorTok{!}\StringTok{'ffdf'}\OperatorTok\KeywordTok{class}\NormalTok{(plpData}\OperatorTok{$}\NormalTok{covariates) )}
    \KeywordTok{stop}\NormalTok{(}\StringTok{'This algorithm requires plpData in coo format'}\NormalTok{)}
  
\NormalTok{  metaData <-}\StringTok{ }\KeywordTok{attr}\NormalTok{(population, }\StringTok{'metaData'}\NormalTok{)}
  \ControlFlowTok{if}\NormalTok{(}\OperatorTok{!}\KeywordTok{is.null}\NormalTok{(population}\OperatorTok{$}\NormalTok{indexes))}
\NormalTok{    population <-}\StringTok{ }\NormalTok{population[population}\OperatorTok{$}\NormalTok{indexes}\OperatorTok{>}\DecValTok{0}\NormalTok{,]}
  \KeywordTok{attr}\NormalTok{(population, }\StringTok{'metaData'}\NormalTok{) <-}\StringTok{ }\NormalTok{metaData}
  
  \CommentTok{# convert data into sparse R Matrix:}
\NormalTok{  result <-}\StringTok{ }\KeywordTok{toSparseM}\NormalTok{(plpData,population,}\DataTypeTok{map=}\OtherTok{NULL}\NormalTok{)}
\NormalTok{  data <-}\StringTok{ }\NormalTok{result}\OperatorTok{$}\NormalTok{data}
  
\NormalTok{  data <-}\StringTok{ }\NormalTok{data[population}\OperatorTok{$}\NormalTok{rowId,]}
  
  \CommentTok{# set test/train sets (for printing performance as it trains)}
  \ControlFlowTok{if}\NormalTok{(}\OperatorTok{!}\NormalTok{quiet)}
    \KeywordTok{writeLines}\NormalTok{(}\KeywordTok{paste0}\NormalTok{(}\StringTok{'Training made up model on train set containing '}\NormalTok{, }\KeywordTok{nrow}\NormalTok{(population), }
                      \StringTok{' people with '}\NormalTok{,}\KeywordTok{sum}\NormalTok{(population}\OperatorTok{$}\NormalTok{outcomeCount}\OperatorTok{>}\DecValTok{0}\NormalTok{), }\StringTok{' outcomes'}\NormalTok{))}
\NormalTok{  start <-}\StringTok{ }\KeywordTok{Sys.time}\NormalTok{()}
  
  \CommentTok{#============= STEP 1 ======================================}
  \CommentTok{# pick the best hyper-params and then do final training on all data...}
  \KeywordTok{writeLines}\NormalTok{(}\StringTok{'train'}\NormalTok{)}
\NormalTok{  datas <-}\StringTok{ }\KeywordTok{list}\NormalTok{(}\DataTypeTok{population=}\NormalTok{population, }\DataTypeTok{data=}\NormalTok{data)}
\NormalTok{  param.sel <-}\StringTok{ }\KeywordTok{lapply}\NormalTok{(param, }\ControlFlowTok{function}\NormalTok{(x) }\KeywordTok{do.call}\NormalTok{(made_up_model, }\KeywordTok{c}\NormalTok{(x,datas)  ))}
\NormalTok{  hyperSummary <-}\StringTok{ }\KeywordTok{do.call}\NormalTok{(rbind, }\KeywordTok{lapply}\NormalTok{(param.sel, }\ControlFlowTok{function}\NormalTok{(x) x}\OperatorTok{$}\NormalTok{hyperSum))}
\NormalTok{  hyperSummary <-}\StringTok{ }\KeywordTok{as.data.frame}\NormalTok{(hyperSummary)}
\NormalTok{  hyperSummary}\OperatorTok{$}\NormalTok{auc <-}\StringTok{ }\KeywordTok{unlist}\NormalTok{(}\KeywordTok{lapply}\NormalTok{(param.sel, }\ControlFlowTok{function}\NormalTok{(x) x}\OperatorTok{$}\NormalTok{auc)) }
\NormalTok{  param.sel <-}\StringTok{ }\KeywordTok{unlist}\NormalTok{(}\KeywordTok{lapply}\NormalTok{(param.sel, }\ControlFlowTok{function}\NormalTok{(x) x}\OperatorTok{$}\NormalTok{auc))}
\NormalTok{  param <-}\StringTok{ }\NormalTok{param[[}\KeywordTok{which.max}\NormalTok{(param.sel)]]}
  
  \CommentTok{# set this so you do a final model train }
\NormalTok{  param}\OperatorTok{$}\NormalTok{final=T}
  
  \KeywordTok{writeLines}\NormalTok{(}\StringTok{'final train'}\NormalTok{)}
\NormalTok{  trainedModel <-}\StringTok{ }\KeywordTok{do.call}\NormalTok{(made_up_model, }\KeywordTok{c}\NormalTok{(param,datas)  )}\OperatorTok{$}\NormalTok{model}
  
\NormalTok{  comp <-}\StringTok{ }\KeywordTok{Sys.time}\NormalTok{() }\OperatorTok{-}\StringTok{ }\NormalTok{start}
  \ControlFlowTok{if}\NormalTok{(}\OperatorTok{!}\NormalTok{quiet)}
    \KeywordTok{writeLines}\NormalTok{(}\KeywordTok{paste0}\NormalTok{(}\StringTok{'Model Made Up trained - took:'}\NormalTok{,  }\KeywordTok{format}\NormalTok{(comp, }\DataTypeTok{digits=}\DecValTok{3}\NormalTok{)))}
  
  \CommentTok{# construct the standard output for a model:}
\NormalTok{  result <-}\StringTok{ }\KeywordTok{list}\NormalTok{(}\DataTypeTok{model =}\NormalTok{ trainedModel,}
                 \DataTypeTok{modelSettings =} \KeywordTok{list}\NormalTok{(}\DataTypeTok{model=}\StringTok{'made_up'}\NormalTok{, }\DataTypeTok{modelParameters=}\NormalTok{param),}
                 \DataTypeTok{trainCVAuc =} \OtherTok{NULL}\NormalTok{,}
                 \DataTypeTok{hyperParamSearch =}\NormalTok{ hyperSummary,}
                 \DataTypeTok{metaData =}\NormalTok{ plpData}\OperatorTok{$}\NormalTok{metaData,}
                 \DataTypeTok{populationSettings =} \KeywordTok{attr}\NormalTok{(population, }\StringTok{'metaData'}\NormalTok{),}
                 \DataTypeTok{outcomeId=}\NormalTok{outcomeId,}\CommentTok{# can use populationSettings$outcomeId?}
                 \DataTypeTok{cohortId=}\NormalTok{cohortId,}
                 \DataTypeTok{varImp =} \OtherTok{NULL}\NormalTok{,}
                 \DataTypeTok{trainingTime=}\NormalTok{comp,}
                 \DataTypeTok{covariateMap=}\NormalTok{result}\OperatorTok{$}\NormalTok{map}
\NormalTok{  )}
  \KeywordTok{class}\NormalTok{(result) <-}\StringTok{ 'plpModel'}
  \KeywordTok{attr}\NormalTok{(result, }\StringTok{'type'}\NormalTok{) <-}\StringTok{ 'madeup'}
  \KeywordTok{attr}\NormalTok{(result, }\StringTok{'predictionType'}\NormalTok{) <-}\StringTok{ 'binary'}
  \KeywordTok{return}\NormalTok{(result)}
    
\NormalTok{\}}
\end{Highlighting}
\end{Shaded}

\subsection{Helpers}\label{helpers}

In the fit model a helper function \texttt{made\_up\_model} is called,
this is the function that trains a model given the data and population
(where the popualtion contains a column outcomeCount corresponding to
the outcome). Both the data and population are ordered the same way:

\begin{Shaded}
\begin{Highlighting}[]
\NormalTok{made_up_model <-}\StringTok{ }\ControlFlowTok{function}\NormalTok{(data, population,}
                       \DataTypeTok{a=}\DecValTok{1}\NormalTok{,}\DataTypeTok{b=}\DecValTok{1}\NormalTok{, }\DataTypeTok{final=}\NormalTok{F, ...)\{}
  
  \KeywordTok{writeLines}\NormalTok{(}\KeywordTok{paste}\NormalTok{(}\StringTok{'Training Made Up model with '}\NormalTok{,}\KeywordTok{length}\NormalTok{(}\KeywordTok{unique}\NormalTok{(population}\OperatorTok{$}\NormalTok{indexes)),}
                   \StringTok{' fold CV'}\NormalTok{))}
  \ControlFlowTok{if}\NormalTok{(}\OperatorTok{!}\KeywordTok{is.null}\NormalTok{(population}\OperatorTok{$}\NormalTok{indexes) }\OperatorTok{&&}\StringTok{ }\NormalTok{final}\OperatorTok{==}\NormalTok{F)\{}
\NormalTok{    index_vect <-}\StringTok{ }\KeywordTok{unique}\NormalTok{(population}\OperatorTok{$}\NormalTok{indexes)}
\NormalTok{    perform <-}\StringTok{ }\KeywordTok{c}\NormalTok{()}
    
    \CommentTok{# create prediction matrix to store all predictions}
\NormalTok{    predictionMat <-}\StringTok{ }\NormalTok{population}
\NormalTok{    predictionMat}\OperatorTok{$}\NormalTok{value <-}\StringTok{ }\DecValTok{0}
    \KeywordTok{attr}\NormalTok{(predictionMat, }\StringTok{"metaData"}\NormalTok{) <-}\StringTok{ }\KeywordTok{list}\NormalTok{(}\DataTypeTok{predictionType =} \StringTok{"binary"}\NormalTok{)}
    
    \ControlFlowTok{for}\NormalTok{(index }\ControlFlowTok{in} \DecValTok{1}\OperatorTok{:}\KeywordTok{length}\NormalTok{(index_vect ))\{}
      \KeywordTok{writeLines}\NormalTok{(}\KeywordTok{paste}\NormalTok{(}\StringTok{'Fold '}\NormalTok{,index, }\StringTok{' -- with '}\NormalTok{, }\KeywordTok{sum}\NormalTok{(population}\OperatorTok{$}\NormalTok{indexes}\OperatorTok{!=}\NormalTok{index),}
                       \StringTok{'train rows'}\NormalTok{))}
\NormalTok{      model <-}\StringTok{ }\NormalTok{madeup}\OperatorTok{::}\KeywordTok{model}\NormalTok{(}\DataTypeTok{x =}\NormalTok{ data[population}\OperatorTok{$}\NormalTok{indexes}\OperatorTok{!=}\NormalTok{index,],}
                             \DataTypeTok{y=}\NormalTok{ population}\OperatorTok{$}\NormalTok{outcomeCount[population}\OperatorTok{$}\NormalTok{indexes}\OperatorTok{!=}\NormalTok{index],}
                                  \DataTypeTok{a=}\NormalTok{a, }\DataTypeTok{b=}\NormalTok{b)}
      
\NormalTok{      pred <-}\StringTok{ }\NormalTok{stats}\OperatorTok{::}\KeywordTok{predict}\NormalTok{(model, data[population}\OperatorTok{$}\NormalTok{indexes}\OperatorTok{==}\NormalTok{index,])}
\NormalTok{      prediction <-}\StringTok{ }\NormalTok{population[population}\OperatorTok{$}\NormalTok{indexes}\OperatorTok{==}\NormalTok{index,]}
\NormalTok{      prediction}\OperatorTok{$}\NormalTok{value <-}\StringTok{ }\NormalTok{pred}
      \KeywordTok{attr}\NormalTok{(prediction, }\StringTok{"metaData"}\NormalTok{) <-}\StringTok{ }\KeywordTok{list}\NormalTok{(}\DataTypeTok{predictionType =} \StringTok{"binary"}\NormalTok{)}
\NormalTok{      aucVal <-}\StringTok{ }\KeywordTok{computeAuc}\NormalTok{(prediction)}
\NormalTok{      perform <-}\StringTok{ }\KeywordTok{c}\NormalTok{(perform,aucVal)}
      
      \CommentTok{# add the fold predictions and compute AUC after loop}
\NormalTok{      predictionMat}\OperatorTok{$}\NormalTok{value[population}\OperatorTok{$}\NormalTok{indexes}\OperatorTok{==}\NormalTok{index] <-}\StringTok{ }\NormalTok{pred}
      
\NormalTok{     \}}
\NormalTok{    ##auc <- mean(perform) # want overal rather than mean}
\NormalTok{    auc <-}\StringTok{ }\KeywordTok{computeAuc}\NormalTok{(predictionMat)}
    
\NormalTok{    foldPerm <-}\StringTok{ }\NormalTok{perform}
\NormalTok{  \} }\ControlFlowTok{else}\NormalTok{ \{}
\NormalTok{    model <-}\StringTok{ }\NormalTok{madeup}\OperatorTok{::}\KeywordTok{model}\NormalTok{(}\DataTypeTok{x=}\NormalTok{ data, }
                                \DataTypeTok{y=}\NormalTok{ population}\OperatorTok{$}\NormalTok{outcomeCount,}
                                \DataTypeTok{a=}\NormalTok{a,}\DataTypeTok{b=}\NormalTok{b)}
    
\NormalTok{    pred <-}\StringTok{ }\NormalTok{stats}\OperatorTok{::}\KeywordTok{predict}\NormalTok{(model, data)}
\NormalTok{    prediction <-}\StringTok{ }\NormalTok{population}
\NormalTok{    prediction}\OperatorTok{$}\NormalTok{value <-}\StringTok{ }\NormalTok{pred}
    \KeywordTok{attr}\NormalTok{(prediction, }\StringTok{"metaData"}\NormalTok{) <-}\StringTok{ }\KeywordTok{list}\NormalTok{(}\DataTypeTok{predictionType =} \StringTok{"binary"}\NormalTok{) }
\NormalTok{    auc <-}\StringTok{ }\KeywordTok{computeAuc}\NormalTok{(prediction)}
\NormalTok{    foldPerm <-}\StringTok{ }\NormalTok{auc}
\NormalTok{  \}}
  
\NormalTok{  result <-}\StringTok{ }\KeywordTok{list}\NormalTok{(}\DataTypeTok{model=}\NormalTok{model,}
                 \DataTypeTok{auc=}\NormalTok{auc,}
                 \DataTypeTok{hyperSum =} \KeywordTok{unlist}\NormalTok{(}\KeywordTok{list}\NormalTok{(}\DataTypeTok{a =}\NormalTok{ a, }\DataTypeTok{b =}\NormalTok{ b, }\DataTypeTok{fold_auc=}\NormalTok{foldPerm))}
\NormalTok{  )}
  \KeywordTok{return}\NormalTok{(result)}
\NormalTok{\}}
\end{Highlighting}
\end{Shaded}

\subsection{Predict}\label{predict-1}

The final step is to create a predict function for the model. This gets
added to the predict.R file. In the example above the type
\texttt{attr(result,\ \textquotesingle{}type\textquotesingle{})\ \textless{}-\ \textquotesingle{}madeup\textquotesingle{}}
was madeup, so a \texttt{predict.madeup} function is required to be
added into the predict.R. The predict function needs to take as input
the plpModel returned by the fit function, the population to apply the
model on and the plpData specifying the covariates of the population.

\begin{Shaded}
\begin{Highlighting}[]
\NormalTok{predict.madeup <-}\StringTok{ }\ControlFlowTok{function}\NormalTok{(plpModel,population, plpData, ...)\{ }
\NormalTok{  result <-}\StringTok{ }\KeywordTok{toSparseM}\NormalTok{(plpData, population, }\DataTypeTok{map=}\NormalTok{plpModel}\OperatorTok{$}\NormalTok{covariateMap)}
\NormalTok{  data <-}\StringTok{ }\NormalTok{result}\OperatorTok{$}\NormalTok{data[population}\OperatorTok{$}\NormalTok{rowId,]}
\NormalTok{  prediction <-}\StringTok{ }\KeywordTok{data.frame}\NormalTok{(}\DataTypeTok{rowId=}\NormalTok{population}\OperatorTok{$}\NormalTok{rowId, }
                           \DataTypeTok{value=}\NormalTok{stats}\OperatorTok{::}\KeywordTok{predict}\NormalTok{(plpModel}\OperatorTok{$}\NormalTok{model, data)}
\NormalTok{                           )}
  
\NormalTok{  prediction <-}\StringTok{ }\KeywordTok{merge}\NormalTok{(population, prediction, }\DataTypeTok{by=}\StringTok{'rowId'}\NormalTok{)}
\NormalTok{  prediction <-}\StringTok{ }\NormalTok{prediction[,}\KeywordTok{colnames}\NormalTok{(prediction)}\OperatorTok
\StringTok{                           }\KeywordTok{c}\NormalTok{(}\StringTok{'rowId'}\NormalTok{,}\StringTok{'outcomeCount'}\NormalTok{,}\StringTok{'indexes'}\NormalTok{, }\StringTok{'value'}\NormalTok{)] }\CommentTok{# need to fix no index issue}
  \KeywordTok{attr}\NormalTok{(prediction, }\StringTok{"metaData"}\NormalTok{) <-}\StringTok{ }\KeywordTok{list}\NormalTok{(}\DataTypeTok{predictionType =} \StringTok{"binary"}\NormalTok{) }
  \KeywordTok{return}\NormalTok{(prediction)}
  
\NormalTok{\}}
\end{Highlighting}
\end{Shaded}

As the madeup model uses the standard R prediction, it has the same
prediction function as xgboost, so we could have not added a new
prediction function and instead made the type of the result returned by
fitMadeUpModel to
\texttt{attr(result,\ \textquotesingle{}type\textquotesingle{})\ \textless{}-\ \textquotesingle{}xgboost\textquotesingle{}}.

\section{Acknowledgments}\label{acknowledgments}

Considerable work has been dedicated to provide the
\texttt{PatientLevelPrediction} package.

\begin{Shaded}
\begin{Highlighting}[]
\KeywordTok{citation}\NormalTok{(}\StringTok{"PatientLevelPrediction"}\NormalTok{)}
\end{Highlighting}
\end{Shaded}

\begin{verbatim}
## 
##   Jenna Reps, Martijn J. Schuemie, Marc A. Suchard, Patrick B.
##   Ryan and Peter R. Rijnbeek (2018). PatientLevelPrediction:
##   Package for patient level prediction using data in the OMOP
##   Common Data Model. R package version 2.0.5.
## 
## A BibTeX entry for LaTeX users is
## 
##   @Manual{,
##     title = {PatientLevelPrediction: Package for patient level prediction using data in the OMOP
## Common Data Model},
##     author = {Jenna Reps and Martijn J. Schuemie and Marc A. Suchard and Patrick B. Ryan and Peter R. Rijnbeek},
##     year = {2018},
##     note = {R package version 2.0.5},
##   }
\end{verbatim}

\textbf{Please reference this paper if you use the PLP Package in your
work:}

\href{http://dx.doi.org/10.1093/jamia/ocy032}{Reps JM, Schuemie MJ,
Suchard MA, Ryan PB, Rijnbeek PR. Design and implementation of a
standardized framework to generate and evaluate patient-level prediction
models using observational healthcare data. J Am Med Inform Assoc.
2018;25(8):969-975.}

This work is supported in part through the National Science Foundation
grant IIS 1251151.


\end{document}
