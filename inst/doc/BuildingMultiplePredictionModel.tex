\documentclass[]{article}
\usepackage{lmodern}
\usepackage{amssymb,amsmath}
\usepackage{ifxetex,ifluatex}
\usepackage{fixltx2e} % provides \textsubscript
\ifnum 0\ifxetex 1\fi\ifluatex 1\fi=0 % if pdftex
  \usepackage[T1]{fontenc}
  \usepackage[utf8]{inputenc}
\else % if luatex or xelatex
  \ifxetex
    \usepackage{mathspec}
  \else
    \usepackage{fontspec}
  \fi
  \defaultfontfeatures{Ligatures=TeX,Scale=MatchLowercase}
\fi
% use upquote if available, for straight quotes in verbatim environments
\IfFileExists{upquote.sty}{\usepackage{upquote}}{}
% use microtype if available
\IfFileExists{microtype.sty}{%
\usepackage{microtype}
\UseMicrotypeSet[protrusion]{basicmath} % disable protrusion for tt fonts
}{}
\usepackage[margin=1in]{geometry}
\usepackage{hyperref}
\hypersetup{unicode=true,
            pdftitle={Building multiple patient-level predictive models},
            pdfauthor={Jenna Reps, Martijn J. Schuemie, Patrick B. Ryan, Peter R. Rijnbeek},
            pdfborder={0 0 0},
            breaklinks=true}
\urlstyle{same}  % don't use monospace font for urls
\usepackage{color}
\usepackage{fancyvrb}
\newcommand{\VerbBar}{|}
\newcommand{\VERB}{\Verb[commandchars=\\\{\}]}
\DefineVerbatimEnvironment{Highlighting}{Verbatim}{commandchars=\\\{\}}
% Add ',fontsize=\small' for more characters per line
\usepackage{framed}
\definecolor{shadecolor}{RGB}{248,248,248}
\newenvironment{Shaded}{\begin{snugshade}}{\end{snugshade}}
\newcommand{\KeywordTok}[1]{\textcolor[rgb]{0.13,0.29,0.53}{\textbf{#1}}}
\newcommand{\DataTypeTok}[1]{\textcolor[rgb]{0.13,0.29,0.53}{#1}}
\newcommand{\DecValTok}[1]{\textcolor[rgb]{0.00,0.00,0.81}{#1}}
\newcommand{\BaseNTok}[1]{\textcolor[rgb]{0.00,0.00,0.81}{#1}}
\newcommand{\FloatTok}[1]{\textcolor[rgb]{0.00,0.00,0.81}{#1}}
\newcommand{\ConstantTok}[1]{\textcolor[rgb]{0.00,0.00,0.00}{#1}}
\newcommand{\CharTok}[1]{\textcolor[rgb]{0.31,0.60,0.02}{#1}}
\newcommand{\SpecialCharTok}[1]{\textcolor[rgb]{0.00,0.00,0.00}{#1}}
\newcommand{\StringTok}[1]{\textcolor[rgb]{0.31,0.60,0.02}{#1}}
\newcommand{\VerbatimStringTok}[1]{\textcolor[rgb]{0.31,0.60,0.02}{#1}}
\newcommand{\SpecialStringTok}[1]{\textcolor[rgb]{0.31,0.60,0.02}{#1}}
\newcommand{\ImportTok}[1]{#1}
\newcommand{\CommentTok}[1]{\textcolor[rgb]{0.56,0.35,0.01}{\textit{#1}}}
\newcommand{\DocumentationTok}[1]{\textcolor[rgb]{0.56,0.35,0.01}{\textbf{\textit{#1}}}}
\newcommand{\AnnotationTok}[1]{\textcolor[rgb]{0.56,0.35,0.01}{\textbf{\textit{#1}}}}
\newcommand{\CommentVarTok}[1]{\textcolor[rgb]{0.56,0.35,0.01}{\textbf{\textit{#1}}}}
\newcommand{\OtherTok}[1]{\textcolor[rgb]{0.56,0.35,0.01}{#1}}
\newcommand{\FunctionTok}[1]{\textcolor[rgb]{0.00,0.00,0.00}{#1}}
\newcommand{\VariableTok}[1]{\textcolor[rgb]{0.00,0.00,0.00}{#1}}
\newcommand{\ControlFlowTok}[1]{\textcolor[rgb]{0.13,0.29,0.53}{\textbf{#1}}}
\newcommand{\OperatorTok}[1]{\textcolor[rgb]{0.81,0.36,0.00}{\textbf{#1}}}
\newcommand{\BuiltInTok}[1]{#1}
\newcommand{\ExtensionTok}[1]{#1}
\newcommand{\PreprocessorTok}[1]{\textcolor[rgb]{0.56,0.35,0.01}{\textit{#1}}}
\newcommand{\AttributeTok}[1]{\textcolor[rgb]{0.77,0.63,0.00}{#1}}
\newcommand{\RegionMarkerTok}[1]{#1}
\newcommand{\InformationTok}[1]{\textcolor[rgb]{0.56,0.35,0.01}{\textbf{\textit{#1}}}}
\newcommand{\WarningTok}[1]{\textcolor[rgb]{0.56,0.35,0.01}{\textbf{\textit{#1}}}}
\newcommand{\AlertTok}[1]{\textcolor[rgb]{0.94,0.16,0.16}{#1}}
\newcommand{\ErrorTok}[1]{\textcolor[rgb]{0.64,0.00,0.00}{\textbf{#1}}}
\newcommand{\NormalTok}[1]{#1}
\usepackage{graphicx,grffile}
\makeatletter
\def\maxwidth{\ifdim\Gin@nat@width>\linewidth\linewidth\else\Gin@nat@width\fi}
\def\maxheight{\ifdim\Gin@nat@height>\textheight\textheight\else\Gin@nat@height\fi}
\makeatother
% Scale images if necessary, so that they will not overflow the page
% margins by default, and it is still possible to overwrite the defaults
% using explicit options in \includegraphics[width, height, ...]{}
\setkeys{Gin}{width=\maxwidth,height=\maxheight,keepaspectratio}
\IfFileExists{parskip.sty}{%
\usepackage{parskip}
}{% else
\setlength{\parindent}{0pt}
\setlength{\parskip}{6pt plus 2pt minus 1pt}
}
\setlength{\emergencystretch}{3em}  % prevent overfull lines
\providecommand{\tightlist}{%
  \setlength{\itemsep}{0pt}\setlength{\parskip}{0pt}}
\setcounter{secnumdepth}{5}
% Redefines (sub)paragraphs to behave more like sections
\ifx\paragraph\undefined\else
\let\oldparagraph\paragraph
\renewcommand{\paragraph}[1]{\oldparagraph{#1}\mbox{}}
\fi
\ifx\subparagraph\undefined\else
\let\oldsubparagraph\subparagraph
\renewcommand{\subparagraph}[1]{\oldsubparagraph{#1}\mbox{}}
\fi

%%% Use protect on footnotes to avoid problems with footnotes in titles
\let\rmarkdownfootnote\footnote%
\def\footnote{\protect\rmarkdownfootnote}

%%% Change title format to be more compact
\usepackage{titling}

% Create subtitle command for use in maketitle
\newcommand{\subtitle}[1]{
  \posttitle{
    \begin{center}\large#1\end{center}
    }
}

\setlength{\droptitle}{-2em}
  \title{Building multiple patient-level predictive models}
  \pretitle{\vspace{\droptitle}\centering\huge}
  \posttitle{\par}
  \author{Jenna Reps, Martijn J. Schuemie, Patrick B. Ryan, Peter R. Rijnbeek}
  \preauthor{\centering\large\emph}
  \postauthor{\par}
  \predate{\centering\large\emph}
  \postdate{\par}
  \date{2018-09-05}


\begin{document}
\maketitle

{
\setcounter{tocdepth}{2}
\tableofcontents
}
\newpage

\section{Introduction}\label{introduction}

This vignette describes how you can use the
\texttt{PatientLevelPrediction} package to build multiple patient-level
predictive models. This vignette assumes you have read and are
comfortable with running single patient level prediction models.

\section{Study Populations}\label{study-populations}

The create a study population setting use the function
\texttt{createStudyPopulationSettings}. If we want to test out three
study populations:

\begin{itemize}
\tightlist
\item
  study population 1 includes people who have the outcome but leave the
  database before the end of time-at-risk and only those without the
  outcome who are observed for the whole time-at-risk period.
\item
  study population 2 includes people who are observed for the whole
  time-at-risk period.
\item
  study population 3 includes people who are not observed for the whole
  time-at-risk
\end{itemize}

we can make all three populations and then combine them into a list:

\begin{Shaded}
\begin{Highlighting}[]
\NormalTok{studyPop1 <-}\StringTok{ }\KeywordTok{createStudyPopulationSettings}\NormalTok{(}\DataTypeTok{binary =}\NormalTok{ T,}
                                          \DataTypeTok{includeAllOutcomes =}\NormalTok{ T,}
                                          \DataTypeTok{removeSubjectsWithPriorOutcome =} \OtherTok{TRUE}\NormalTok{,}
                                          \DataTypeTok{priorOutcomeLookback =} \DecValTok{99999}\NormalTok{,}
                                          \DataTypeTok{requireTimeAtRisk =}\NormalTok{ T,}
                                          \DataTypeTok{minTimeAtRisk=}\DecValTok{364}\NormalTok{,}
                                          \DataTypeTok{riskWindowStart =} \DecValTok{1}\NormalTok{,}
                                          \DataTypeTok{riskWindowEnd =} \DecValTok{365}\NormalTok{,}
                                          \DataTypeTok{verbosity =} \StringTok{"INFO"}\NormalTok{)}
\NormalTok{studyPop2 <-}\StringTok{ }\KeywordTok{createStudyPopulationSettings}\NormalTok{(}\DataTypeTok{binary =}\NormalTok{ T,}
                                           \DataTypeTok{includeAllOutcomes =}\NormalTok{ F,}
                                           \DataTypeTok{removeSubjectsWithPriorOutcome =} \OtherTok{TRUE}\NormalTok{,}
                                           \DataTypeTok{priorOutcomeLookback =} \DecValTok{99999}\NormalTok{,}
                                           \DataTypeTok{requireTimeAtRisk =}\NormalTok{ T,}
                                           \DataTypeTok{minTimeAtRisk=}\DecValTok{364}\NormalTok{,}
                                           \DataTypeTok{riskWindowStart =} \DecValTok{1}\NormalTok{,}
                                           \DataTypeTok{riskWindowEnd =} \DecValTok{365}\NormalTok{,}
                                           \DataTypeTok{verbosity =} \StringTok{"INFO"}\NormalTok{)}
\NormalTok{studyPop3 <-}\StringTok{ }\KeywordTok{createStudyPopulationSettings}\NormalTok{(}\DataTypeTok{binary =}\NormalTok{ T,}
                                           \DataTypeTok{includeAllOutcomes =}\NormalTok{ F,}
                                           \DataTypeTok{removeSubjectsWithPriorOutcome =} \OtherTok{TRUE}\NormalTok{,}
                                           \DataTypeTok{priorOutcomeLookback =} \DecValTok{99999}\NormalTok{,}
                                           \DataTypeTok{requireTimeAtRisk =}\NormalTok{ F,}
                                           \DataTypeTok{minTimeAtRisk=}\DecValTok{364}\NormalTok{,}
                                           \DataTypeTok{riskWindowStart =} \DecValTok{1}\NormalTok{,}
                                           \DataTypeTok{riskWindowEnd =} \DecValTok{365}\NormalTok{,}
                                           \DataTypeTok{verbosity =} \StringTok{"INFO"}\NormalTok{)}
                                           
\NormalTok{populationSettingList <-}\StringTok{ }\KeywordTok{list}\NormalTok{(studyPop1,studyPop2,studyPop3)}
\end{Highlighting}
\end{Shaded}

\section{Covariate Settings}\label{covariate-settings}

The covariate settings are created using
\texttt{createCovariateSettings}. We create seperate covariate settings
and then combine them into a list:

\begin{Shaded}
\begin{Highlighting}[]
\NormalTok{covSet1 <-}\StringTok{ }\KeywordTok{createCovariateSettings}\NormalTok{(}\DataTypeTok{useDemographicsGender =}\NormalTok{ T, }
                                   \DataTypeTok{useDemographicsAgeGroup =}\NormalTok{ T, }
                                   \DataTypeTok{useConditionGroupEraAnyTimePrior =}\NormalTok{ T,}
                                   \DataTypeTok{useDrugGroupEraAnyTimePrior =}\NormalTok{ T)}
\NormalTok{covSet2 <-}\StringTok{ }\KeywordTok{createCovariateSettings}\NormalTok{(}\DataTypeTok{useDemographicsGender =}\NormalTok{ T, }
                                   \DataTypeTok{useDemographicsAgeGroup =}\NormalTok{ T, }
                                   \DataTypeTok{useConditionGroupEraAnyTimePrior =}\NormalTok{ T,}
                                   \DataTypeTok{useDrugGroupEraAnyTimePrior =}\NormalTok{ F)}
\NormalTok{covariateSettingList <-}\StringTok{ }\KeywordTok{list}\NormalTok{(covSet1, covSet2)}
\end{Highlighting}
\end{Shaded}

\section{Model Settings}\label{model-settings}

The model settings requires running the setModel functions for the
machine learning models of interest and specifying the hyper-parameter
search and then combining these into a list. For example, if we wanted
to try a logistic regression, gradient boosting machine and ada boost
model then:

\begin{Shaded}
\begin{Highlighting}[]
\NormalTok{gbm <-}\StringTok{ }\KeywordTok{setGradientBoostingMachine}\NormalTok{()}
\NormalTok{lr <-}\StringTok{ }\KeywordTok{setLassoLogisticRegression}\NormalTok{()}
\NormalTok{ada <-}\StringTok{ }\KeywordTok{setAdaBoost}\NormalTok{()}

\NormalTok{modelList <-}\StringTok{ }\KeywordTok{list}\NormalTok{(gbm, lr, ada)}
\end{Highlighting}
\end{Shaded}

\section{Creating the model analysis
list}\label{creating-the-model-analysis-list}

To create the complete plp model settings use
\texttt{createPlpModelSettings} to combine the population, covariate and
model settings.

\begin{Shaded}
\begin{Highlighting}[]
\NormalTok{modelAnalysisList <-}\StringTok{ }\KeywordTok{createPlpModelSettings}\NormalTok{(}\DataTypeTok{modelList =}\NormalTok{ modelList, }
                                   \DataTypeTok{covariateSettingList =}\NormalTok{ covariateSettingList,}
                                   \DataTypeTok{populationSettingList =}\NormalTok{ populationSettingList)}
\end{Highlighting}
\end{Shaded}

\section{Running the multiple prediction
patient-level-prediction}\label{running-the-multiple-prediction-patient-level-prediction}

As we will be downloading loads of data in the multiple plp analysis it
is useful to set fftempdir to a directory with write access and plenty
of space.
\texttt{options(fftempdir\ =\ \textquotesingle{}T:/fftemp\textquotesingle{})}

To run the study requires setting up a connectionDetails object

\begin{Shaded}
\begin{Highlighting}[]
\NormalTok{dbms <-}\StringTok{ "your dbms"}
\NormalTok{user <-}\StringTok{ "your username"}
\NormalTok{pw <-}\StringTok{ "your password"}
\NormalTok{server <-}\StringTok{ "your server"}
\NormalTok{port <-}\StringTok{ "your port"}

\NormalTok{connectionDetails <-}\StringTok{ }\NormalTok{DatabaseConnector}\OperatorTok{::}\KeywordTok{createConnectionDetails}\NormalTok{(}\DataTypeTok{dbms =}\NormalTok{ dbms,}
                                                                \DataTypeTok{server =}\NormalTok{ server,}
                                                                \DataTypeTok{user =}\NormalTok{ user,}
                                                                \DataTypeTok{password =}\NormalTok{ pw,}
                                                                \DataTypeTok{port =}\NormalTok{ port)}
\end{Highlighting}
\end{Shaded}

Next you need to specify the cdmDatabaseSchema where your cdm database
is found and workDatabaseSchema where your target population and outcome
cohorts are.

\begin{Shaded}
\begin{Highlighting}[]
\NormalTok{cdmDatabaseSchema <-}\StringTok{ "your cdmDatabaseSchema"}
\NormalTok{workDatabaseSchema <-}\StringTok{ "your workDatabaseSchema"}
\end{Highlighting}
\end{Shaded}

Now you can run the multiple patient-level prediction analysis by
specifying the target cohort ids and outcome ids

\begin{Shaded}
\begin{Highlighting}[]
\NormalTok{allresults <-}\StringTok{ }\KeywordTok{runPlpAnalyses}\NormalTok{(}\DataTypeTok{connectionDetails =}\NormalTok{ connectionDetails,}
                           \DataTypeTok{cdmDatabaseSchema =}\NormalTok{ cdmDatabaseSchema,}
                           \DataTypeTok{oracleTempSchema =}\NormalTok{ cdmDatabaseSchema,}
                           \DataTypeTok{cohortDatabaseSchema =}\NormalTok{ workDatabaseSchema,}
                           \DataTypeTok{cohortTable =} \StringTok{"your cohort table"}\NormalTok{,}
                           \DataTypeTok{outcomeDatabaseSchema =}\NormalTok{ workDatabaseSchema,}
                           \DataTypeTok{outcomeTable =} \StringTok{"your cohort table"}\NormalTok{,}
                           \DataTypeTok{cdmVersion =} \DecValTok{5}\NormalTok{,}
                           \DataTypeTok{outputFolder =} \StringTok{"./PlpMultiOutput"}\NormalTok{,}
                           \DataTypeTok{modelAnalysisList =}\NormalTok{ modelAnalysisList,}
                           \DataTypeTok{cohortIds =} \KeywordTok{c}\NormalTok{(}\DecValTok{2484}\NormalTok{,}\DecValTok{6970}\NormalTok{),}
                           \DataTypeTok{cohortNames =} \KeywordTok{c}\NormalTok{(}\StringTok{'visit 2010'}\NormalTok{,}\StringTok{'test cohort'}\NormalTok{),}
                           \DataTypeTok{outcomeIds =} \KeywordTok{c}\NormalTok{(}\DecValTok{7331}\NormalTok{,}\DecValTok{5287}\NormalTok{),}
                           \DataTypeTok{outcomeNames =}  \KeywordTok{c}\NormalTok{(}\StringTok{'outcome 1'}\NormalTok{,}\StringTok{'outcome 2'}\NormalTok{),}
                           \DataTypeTok{maxSampleSize =} \OtherTok{NULL}\NormalTok{,}
                           \DataTypeTok{minCovariateFraction =} \DecValTok{0}\NormalTok{,}
                           \DataTypeTok{normalizeData =}\NormalTok{ T,}
                           \DataTypeTok{testSplit =} \StringTok{"person"}\NormalTok{,}
                           \DataTypeTok{testFraction =} \FloatTok{0.25}\NormalTok{,}
                           \DataTypeTok{splitSeed =} \OtherTok{NULL}\NormalTok{,}
                           \DataTypeTok{nfold =} \DecValTok{3}\NormalTok{,}
                           \DataTypeTok{verbosity =} \StringTok{"INFO"}\NormalTok{)}
\end{Highlighting}
\end{Shaded}

This will then save all the plpData objects from the study into
``./PlpMultiOutput/plpData'', the populations for the analysis into
``./PlpMultiOutput/population'' and the results into
``./PlpMultiOutput/Result''. The csv named settings.csv found in
``./PlpMultiOutput'' has a row for each prediction model developed and
points to the plpData and popualtion used for the model development, it
also has descriptions of the cohorts and settings if these are input by
the user.

\section{Validating the multiple prediction patient-level-prediction
results}\label{validating-the-multiple-prediction-patient-level-prediction-results}

To validate all the models on new data and cohorts run (where
analysesLocation is the output location you input for runPlpAnalyses()
and the suggested outputLocation is the subdirectory `validation' in
this location):

\begin{Shaded}
\begin{Highlighting}[]
\NormalTok{val <-}\StringTok{ }\KeywordTok{evaluateMultiplePlp}\NormalTok{(}\DataTypeTok{analysesLocation =} \StringTok{"./PlpMultiOutput"}\NormalTok{,}
                                \DataTypeTok{outputLocation =} \StringTok{"./PlpMultiOutput/validation"}\NormalTok{,}
                                \DataTypeTok{connectionDetails =}\NormalTok{ connectionDetails, }
                                \DataTypeTok{validationSchemaTarget =} \KeywordTok{list}\NormalTok{(}\StringTok{'new_database_1.dbo'}\NormalTok{,}
                                                              \StringTok{'new_database_2.dbo'}\NormalTok{),}
                                \DataTypeTok{validationSchemaOutcome =} \KeywordTok{list}\NormalTok{(}\StringTok{'new_database_1.dbo'}\NormalTok{,}
                                                              \StringTok{'new_database_2.dbo'}\NormalTok{),}
                                \DataTypeTok{validationSchemaCdm =} \KeywordTok{list}\NormalTok{(}\StringTok{'new_database_1.dbo'}\NormalTok{,}
                                                              \StringTok{'new_database_2.dbo'}\NormalTok{), }
                                \DataTypeTok{databaseNames =} \KeywordTok{c}\NormalTok{(}\StringTok{'database1'}\NormalTok{,}\StringTok{'database2'}\NormalTok{),}
                                \DataTypeTok{validationTableTarget =} \StringTok{'your new cohort table'}\NormalTok{,}
                                \DataTypeTok{validationTableOutcome =} \StringTok{'your new cohort table'}\NormalTok{)}
\end{Highlighting}
\end{Shaded}

This then saves the external validation results in the validation folder
of the main study (the outputLocation you used in runPlpAnalyses).

\section{Viewing the multiple prediction patient-level-prediction
results}\label{viewing-the-multiple-prediction-patient-level-prediction-results}

To view the results for the multiple prediction analysis:

\begin{Shaded}
\begin{Highlighting}[]
\KeywordTok{viewMultiplePlp}\NormalTok{(}\DataTypeTok{analysesLocation=}\StringTok{"./PlpMultiOutput"}\NormalTok{)}
\end{Highlighting}
\end{Shaded}

If the validation directory in ``./PlpMultiOutput'' has results, the
external validation will also be displayed.


\end{document}
