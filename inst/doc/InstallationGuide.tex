\documentclass[]{article}
\usepackage{lmodern}
\usepackage{amssymb,amsmath}
\usepackage{ifxetex,ifluatex}
\usepackage{fixltx2e} % provides \textsubscript
\ifnum 0\ifxetex 1\fi\ifluatex 1\fi=0 % if pdftex
  \usepackage[T1]{fontenc}
  \usepackage[utf8]{inputenc}
  \usepackage{eurosym}
\else % if luatex or xelatex
  \ifxetex
    \usepackage{mathspec}
  \else
    \usepackage{fontspec}
  \fi
  \defaultfontfeatures{Ligatures=TeX,Scale=MatchLowercase}
  \newcommand{\euro}{€}
\fi
% use upquote if available, for straight quotes in verbatim environments
\IfFileExists{upquote.sty}{\usepackage{upquote}}{}
% use microtype if available
\IfFileExists{microtype.sty}{%
\usepackage{microtype}
\UseMicrotypeSet[protrusion]{basicmath} % disable protrusion for tt fonts
}{}
\usepackage[margin=1in]{geometry}
\usepackage{hyperref}
\hypersetup{unicode=true,
            pdftitle={Patient-level prediction installation guide},
            pdfauthor={Jenna Reps},
            pdfborder={0 0 0},
            breaklinks=true}
\urlstyle{same}  % don't use monospace font for urls
\usepackage{color}
\usepackage{fancyvrb}
\newcommand{\VerbBar}{|}
\newcommand{\VERB}{\Verb[commandchars=\\\{\}]}
\DefineVerbatimEnvironment{Highlighting}{Verbatim}{commandchars=\\\{\}}
% Add ',fontsize=\small' for more characters per line
\usepackage{framed}
\definecolor{shadecolor}{RGB}{248,248,248}
\newenvironment{Shaded}{\begin{snugshade}}{\end{snugshade}}
\newcommand{\KeywordTok}[1]{\textcolor[rgb]{0.13,0.29,0.53}{\textbf{#1}}}
\newcommand{\DataTypeTok}[1]{\textcolor[rgb]{0.13,0.29,0.53}{#1}}
\newcommand{\DecValTok}[1]{\textcolor[rgb]{0.00,0.00,0.81}{#1}}
\newcommand{\BaseNTok}[1]{\textcolor[rgb]{0.00,0.00,0.81}{#1}}
\newcommand{\FloatTok}[1]{\textcolor[rgb]{0.00,0.00,0.81}{#1}}
\newcommand{\ConstantTok}[1]{\textcolor[rgb]{0.00,0.00,0.00}{#1}}
\newcommand{\CharTok}[1]{\textcolor[rgb]{0.31,0.60,0.02}{#1}}
\newcommand{\SpecialCharTok}[1]{\textcolor[rgb]{0.00,0.00,0.00}{#1}}
\newcommand{\StringTok}[1]{\textcolor[rgb]{0.31,0.60,0.02}{#1}}
\newcommand{\VerbatimStringTok}[1]{\textcolor[rgb]{0.31,0.60,0.02}{#1}}
\newcommand{\SpecialStringTok}[1]{\textcolor[rgb]{0.31,0.60,0.02}{#1}}
\newcommand{\ImportTok}[1]{#1}
\newcommand{\CommentTok}[1]{\textcolor[rgb]{0.56,0.35,0.01}{\textit{#1}}}
\newcommand{\DocumentationTok}[1]{\textcolor[rgb]{0.56,0.35,0.01}{\textbf{\textit{#1}}}}
\newcommand{\AnnotationTok}[1]{\textcolor[rgb]{0.56,0.35,0.01}{\textbf{\textit{#1}}}}
\newcommand{\CommentVarTok}[1]{\textcolor[rgb]{0.56,0.35,0.01}{\textbf{\textit{#1}}}}
\newcommand{\OtherTok}[1]{\textcolor[rgb]{0.56,0.35,0.01}{#1}}
\newcommand{\FunctionTok}[1]{\textcolor[rgb]{0.00,0.00,0.00}{#1}}
\newcommand{\VariableTok}[1]{\textcolor[rgb]{0.00,0.00,0.00}{#1}}
\newcommand{\ControlFlowTok}[1]{\textcolor[rgb]{0.13,0.29,0.53}{\textbf{#1}}}
\newcommand{\OperatorTok}[1]{\textcolor[rgb]{0.81,0.36,0.00}{\textbf{#1}}}
\newcommand{\BuiltInTok}[1]{#1}
\newcommand{\ExtensionTok}[1]{#1}
\newcommand{\PreprocessorTok}[1]{\textcolor[rgb]{0.56,0.35,0.01}{\textit{#1}}}
\newcommand{\AttributeTok}[1]{\textcolor[rgb]{0.77,0.63,0.00}{#1}}
\newcommand{\RegionMarkerTok}[1]{#1}
\newcommand{\InformationTok}[1]{\textcolor[rgb]{0.56,0.35,0.01}{\textbf{\textit{#1}}}}
\newcommand{\WarningTok}[1]{\textcolor[rgb]{0.56,0.35,0.01}{\textbf{\textit{#1}}}}
\newcommand{\AlertTok}[1]{\textcolor[rgb]{0.94,0.16,0.16}{#1}}
\newcommand{\ErrorTok}[1]{\textcolor[rgb]{0.64,0.00,0.00}{\textbf{#1}}}
\newcommand{\NormalTok}[1]{#1}
\usepackage{graphicx,grffile}
\makeatletter
\def\maxwidth{\ifdim\Gin@nat@width>\linewidth\linewidth\else\Gin@nat@width\fi}
\def\maxheight{\ifdim\Gin@nat@height>\textheight\textheight\else\Gin@nat@height\fi}
\makeatother
% Scale images if necessary, so that they will not overflow the page
% margins by default, and it is still possible to overwrite the defaults
% using explicit options in \includegraphics[width, height, ...]{}
\setkeys{Gin}{width=\maxwidth,height=\maxheight,keepaspectratio}
\IfFileExists{parskip.sty}{%
\usepackage{parskip}
}{% else
\setlength{\parindent}{0pt}
\setlength{\parskip}{6pt plus 2pt minus 1pt}
}
\setlength{\emergencystretch}{3em}  % prevent overfull lines
\providecommand{\tightlist}{%
  \setlength{\itemsep}{0pt}\setlength{\parskip}{0pt}}
\setcounter{secnumdepth}{5}
% Redefines (sub)paragraphs to behave more like sections
\ifx\paragraph\undefined\else
\let\oldparagraph\paragraph
\renewcommand{\paragraph}[1]{\oldparagraph{#1}\mbox{}}
\fi
\ifx\subparagraph\undefined\else
\let\oldsubparagraph\subparagraph
\renewcommand{\subparagraph}[1]{\oldsubparagraph{#1}\mbox{}}
\fi

%%% Use protect on footnotes to avoid problems with footnotes in titles
\let\rmarkdownfootnote\footnote%
\def\footnote{\protect\rmarkdownfootnote}

%%% Change title format to be more compact
\usepackage{titling}

% Create subtitle command for use in maketitle
\newcommand{\subtitle}[1]{
  \posttitle{
    \begin{center}\large#1\end{center}
    }
}

\setlength{\droptitle}{-2em}
  \title{Patient-level prediction installation guide}
  \pretitle{\vspace{\droptitle}\centering\huge}
  \posttitle{\par}
  \author{Jenna Reps}
  \preauthor{\centering\large\emph}
  \postauthor{\par}
  \predate{\centering\large\emph}
  \postdate{\par}
  \date{2018-05-30}


\begin{document}
\maketitle

{
\setcounter{tocdepth}{2}
\tableofcontents
}
\newpage

\section{Introduction}\label{introduction}

\subsection{Windows Users}\label{windows-users}

The OHDSI Patient Level Prediction (PLP) package requires installing:

\begin{itemize}
\tightlist
\item
  R (\url{https://cran.cnr.berkeley.edu/} ) - the latest version is
  recommended
\item
  Rstudio (\url{https://www.rstudio.com/} )
\item
  Java (\url{http://www.java.com} )
\item
  RTools (\url{https://cran.r-project.org/bin/windows/Rtools/})
\item
  Anaconda 3.6 (\url{https://www.continuum.io/downloads}) - this will
  require checking your path variable to ensure the correct python is
  used by R - more instructions below
\end{itemize}

\hypertarget{maclinux-users}{\subsection{Mac/Linux
Users}\label{maclinux-users}}

The OHDSI Patient Level Prediction (PLP) package requires installing:

\begin{itemize}
\tightlist
\item
  R (\url{https://cran.cnr.berkeley.edu/} ) - the latest version is
  recommended
\item
  Rstudio (\url{https://www.rstudio.com/} )
\item
  Java (\url{http://www.java.com} )
\item
  Xcode command line tools(run in terminal: xcode-select --install)
  {[}MAC USERS ONLY{]}
\item
  Python 3.6 (\url{https://www.python.org/downloads/}) - this will
  require checking your path variable to ensure this version python is
  added - more instructions below
\end{itemize}

\subsubsection{Setting up Python for Mac/Linux
Users}\label{setting-up-python-for-maclinux-users}

After installing python 3.6 you can type python3 to open python in a
terminal.

To get the package dependencies, in a terminal run:

\begin{verbatim}
pip3 install --upgrade pip
pip3 install -U NumPy
pip3 install -U SciPy 
pip3 install -U scikit-learn
pip3 install --upgrade tensorflow 
pip3 install keras
\end{verbatim}

\section{Checking Python in Path}\label{checking-python-in-path}

Both anaconda and python 3.6 change the path by default. You can access
the path variable in R using
\texttt{Sys.getenv(\textquotesingle{}PATH\textquotesingle{})}. This
should contain the location of your anaconda or python 3.6.

If you cannot find the anaconda/python in your path, then you need to
add it:

\subsection{Windows users need to:}\label{windows-users-need-to}

my computer -\textgreater{} system properties -\textgreater{} advanced
system settings Then at the bottom right you'll see a button:
Environmental Variables, clicking on that will enable you yo edit the
PATH variable to add the anaconda location.

\subsection{Mac/Linux users need to:}\label{maclinux-users-need-to}

Need to edit the bash profile to add it by running in the terminal:
\texttt{touch\ \textasciitilde{}/.bash\_profile;\ open\ \textasciitilde{}/.bash\_profile;}
and adding in the location of python 3.6.

Please note: if you update the path while R is open, you will need to
shut R down and reopen before the path is refreshed.

\section{Installing PatientLevelPrediction using
drat}\label{installing-patientlevelprediction-using-drat}

\begin{Shaded}
\begin{Highlighting}[]
\KeywordTok{install.packages}\NormalTok{(}\StringTok{"drat"}\NormalTok{)}
\NormalTok{drat}\OperatorTok{::}\KeywordTok{addRepo}\NormalTok{(}\StringTok{"OHDSI"}\NormalTok{)}
\KeywordTok{install.packages}\NormalTok{(}\StringTok{"PatientLevelPrediction"}\NormalTok{)}
\CommentTok{# to get the latest PLP then use devtools}
\KeywordTok{install.packages}\NormalTok{(}\StringTok{"devtools"}\NormalTok{)}
\KeywordTok{install_github}\NormalTok{(}\StringTok{"ohdsi/PatientLevelPrediction"}\NormalTok{)}
\end{Highlighting}
\end{Shaded}

\section{Installing PatientLevelPrediction using
devtools}\label{installing-patientlevelprediction-using-devtools}

If the drat code fails as the repository is not available you can try:

\begin{Shaded}
\begin{Highlighting}[]
\KeywordTok{install.packages}\NormalTok{(}\StringTok{"devtools"}\NormalTok{)}
\KeywordTok{library}\NormalTok{(}\StringTok{"devtools"}\NormalTok{)}
\KeywordTok{install_github}\NormalTok{(}\StringTok{"ohdsi/SqlRender"}\NormalTok{)}
\KeywordTok{install_github}\NormalTok{(}\StringTok{"ohdsi/DatabaseConnectorJars"}\NormalTok{)}
\KeywordTok{install_github}\NormalTok{(}\StringTok{"ohdsi/DatabaseConnector"}\NormalTok{)}
\KeywordTok{install_github}\NormalTok{(}\StringTok{"ohdsi/FeatureExtraction"}\NormalTok{)}
\KeywordTok{install_github}\NormalTok{(}\StringTok{"ohdsi/OhdsiSharing"}\NormalTok{)}
\KeywordTok{install_github}\NormalTok{(}\StringTok{"ohdsi/OhdsiRTools"}\NormalTok{)}
\KeywordTok{install_github}\NormalTok{(}\StringTok{"ohdsi/BigKnn"}\NormalTok{)}
\KeywordTok{install_github}\NormalTok{(}\StringTok{"ohdsi/PatientLevelPrediction"}\NormalTok{)}
\end{Highlighting}
\end{Shaded}

\section{\texorpdfstring{Configuring PythonInR
\protect\hyperlink{maclinux-users}{Mac/Linux
users}}{Configuring PythonInR Mac/Linux users}}\label{configuring-pythoninr-maclinux-users}

Non-windows users need to specify their R environment to get R to use
python 3.6 rather than the default python. In a new Rstudio session run
this to open the environment file:

\begin{Shaded}
\begin{Highlighting}[]
\KeywordTok{install.packages}\NormalTok{(‘usethis’)}
\NormalTok{usethis}\OperatorTok{::}\KeywordTok{edit_r_environ}\NormalTok{()}\OperatorTok{::}\KeywordTok{edit_r_environ}\NormalTok{()}
\end{Highlighting}
\end{Shaded}

In the file that opens add and save: PATH= \{The path containing the
python 3\} USESPECIALPYTHONVERSION=â\euro{}œpython3.6"

For example mine was:
PATH=``/Library/Frameworks/Python.framework/Versions/3.6/bin:/usr/local/bin:/usr/bin:/bin:/usr/sbin:/sbin:/Library/TeX/texbin''
USESPECIALPYTHONVERSION=``python3.6''

You now need to compile PythonInR so it uses python 3.6. In a new R
studio session run:

\begin{Shaded}
\begin{Highlighting}[]
\KeywordTok{Sys.setenv}\NormalTok{(}\DataTypeTok{USESPECIALPYTHONVERSION =} \StringTok{"python3.6"}\NormalTok{)}
\NormalTok{devtools}\OperatorTok{::}\KeywordTok{install_bitbucket}\NormalTok{(}\StringTok{"Floooo/PythonInR"}\NormalTok{)}
\end{Highlighting}
\end{Shaded}

This should now set the PythonInR package to use your python 3.6.

\section{Testing installation}\label{testing-installation}

To test the installation (including python) run:

\begin{Shaded}
\begin{Highlighting}[]
\KeywordTok{library}\NormalTok{(DatabaseConnector)}
\NormalTok{connectionDetails <-}\StringTok{ }\KeywordTok{createConnectionDetails}\NormalTok{(}\DataTypeTok{dbms =} \StringTok{"sql_server"}\NormalTok{, }\DataTypeTok{user =} \StringTok{"username"}\NormalTok{, }
    \DataTypeTok{password =} \StringTok{"hidden"}\NormalTok{, }\DataTypeTok{server =} \StringTok{"your server"}\NormalTok{, }\DataTypeTok{port =} \StringTok{"your port"}\NormalTok{)}
\NormalTok{PatientLevelPrediction}\OperatorTok{::}\KeywordTok{checkPlpInstallation}\NormalTok{(}\DataTypeTok{connectionDetails =}\NormalTok{ connectionDetails, }
    \DataTypeTok{python =}\NormalTok{ T)}
\end{Highlighting}
\end{Shaded}

To test the installation (excluding python) run:

\begin{Shaded}
\begin{Highlighting}[]
\KeywordTok{library}\NormalTok{(DatabaseConnector)}
\NormalTok{connectionDetails <-}\StringTok{ }\KeywordTok{createConnectionDetails}\NormalTok{(}\DataTypeTok{dbms =} \StringTok{"sql_server"}\NormalTok{, }\DataTypeTok{user =} \StringTok{"username"}\NormalTok{, }
    \DataTypeTok{password =} \StringTok{"hidden"}\NormalTok{, }\DataTypeTok{server =} \StringTok{"your server"}\NormalTok{, }\DataTypeTok{port =} \StringTok{"your port"}\NormalTok{)}
\NormalTok{PatientLevelPrediction}\OperatorTok{::}\KeywordTok{checkPlpInstallation}\NormalTok{(}\DataTypeTok{connectionDetails =}\NormalTok{ connectionDetails, }
    \DataTypeTok{python =}\NormalTok{ F)}
\end{Highlighting}
\end{Shaded}

If the above returns a value other than 1 then there is an issue. You
can determine the issue by running:
\texttt{PatientLevelPrediction::interpretInstallCode(N)} where
\texttt{N} is the value returned by the plp installation check.

\section{Adding Keras for deep
learning}\label{adding-keras-for-deep-learning}

To add the R keras interface, in Rstudio run:

\begin{Shaded}
\begin{Highlighting}[]
\NormalTok{devtools}\OperatorTok{::}\KeywordTok{install_github}\NormalTok{(}\StringTok{"rstudio/keras"}\NormalTok{)}
\KeywordTok{library}\NormalTok{(keras)}
\KeywordTok{install_keras}\NormalTok{()}
\end{Highlighting}
\end{Shaded}

\section{Common install issues}\label{common-install-issues}

This will be updated over time.

If you have an error when trying to install a package in R saying
`Dependancy X not available \ldots{}' then this cna sometimes be fixed
by running
\texttt{install.packages(\textquotesingle{}X\textquotesingle{})} and
then once that completes trying to reinstall the package that had the
error.

I have found that using the github devtools to install packages can be
impacted if you have multiple R sessions open as one session with a
library open cause cause the library to be looked and this can prevent
an install of a package that depends on that library.


\end{document}
