\documentclass[]{article}
\usepackage{lmodern}
\usepackage{amssymb,amsmath}
\usepackage{ifxetex,ifluatex}
\usepackage{fixltx2e} % provides \textsubscript
\ifnum 0\ifxetex 1\fi\ifluatex 1\fi=0 % if pdftex
  \usepackage[T1]{fontenc}
  \usepackage[utf8]{inputenc}
\else % if luatex or xelatex
  \ifxetex
    \usepackage{mathspec}
  \else
    \usepackage{fontspec}
  \fi
  \defaultfontfeatures{Ligatures=TeX,Scale=MatchLowercase}
\fi
% use upquote if available, for straight quotes in verbatim environments
\IfFileExists{upquote.sty}{\usepackage{upquote}}{}
% use microtype if available
\IfFileExists{microtype.sty}{%
\usepackage{microtype}
\UseMicrotypeSet[protrusion]{basicmath} % disable protrusion for tt fonts
}{}
\usepackage[margin=1in]{geometry}
\usepackage{hyperref}
\hypersetup{unicode=true,
            pdftitle={Patient-Level Prediction Installation Guide},
            pdfauthor={Jenna Reps, Peter R. Rijnbeek},
            pdfborder={0 0 0},
            breaklinks=true}
\urlstyle{same}  % don't use monospace font for urls
\usepackage{color}
\usepackage{fancyvrb}
\newcommand{\VerbBar}{|}
\newcommand{\VERB}{\Verb[commandchars=\\\{\}]}
\DefineVerbatimEnvironment{Highlighting}{Verbatim}{commandchars=\\\{\}}
% Add ',fontsize=\small' for more characters per line
\usepackage{framed}
\definecolor{shadecolor}{RGB}{248,248,248}
\newenvironment{Shaded}{\begin{snugshade}}{\end{snugshade}}
\newcommand{\KeywordTok}[1]{\textcolor[rgb]{0.13,0.29,0.53}{\textbf{#1}}}
\newcommand{\DataTypeTok}[1]{\textcolor[rgb]{0.13,0.29,0.53}{#1}}
\newcommand{\DecValTok}[1]{\textcolor[rgb]{0.00,0.00,0.81}{#1}}
\newcommand{\BaseNTok}[1]{\textcolor[rgb]{0.00,0.00,0.81}{#1}}
\newcommand{\FloatTok}[1]{\textcolor[rgb]{0.00,0.00,0.81}{#1}}
\newcommand{\ConstantTok}[1]{\textcolor[rgb]{0.00,0.00,0.00}{#1}}
\newcommand{\CharTok}[1]{\textcolor[rgb]{0.31,0.60,0.02}{#1}}
\newcommand{\SpecialCharTok}[1]{\textcolor[rgb]{0.00,0.00,0.00}{#1}}
\newcommand{\StringTok}[1]{\textcolor[rgb]{0.31,0.60,0.02}{#1}}
\newcommand{\VerbatimStringTok}[1]{\textcolor[rgb]{0.31,0.60,0.02}{#1}}
\newcommand{\SpecialStringTok}[1]{\textcolor[rgb]{0.31,0.60,0.02}{#1}}
\newcommand{\ImportTok}[1]{#1}
\newcommand{\CommentTok}[1]{\textcolor[rgb]{0.56,0.35,0.01}{\textit{#1}}}
\newcommand{\DocumentationTok}[1]{\textcolor[rgb]{0.56,0.35,0.01}{\textbf{\textit{#1}}}}
\newcommand{\AnnotationTok}[1]{\textcolor[rgb]{0.56,0.35,0.01}{\textbf{\textit{#1}}}}
\newcommand{\CommentVarTok}[1]{\textcolor[rgb]{0.56,0.35,0.01}{\textbf{\textit{#1}}}}
\newcommand{\OtherTok}[1]{\textcolor[rgb]{0.56,0.35,0.01}{#1}}
\newcommand{\FunctionTok}[1]{\textcolor[rgb]{0.00,0.00,0.00}{#1}}
\newcommand{\VariableTok}[1]{\textcolor[rgb]{0.00,0.00,0.00}{#1}}
\newcommand{\ControlFlowTok}[1]{\textcolor[rgb]{0.13,0.29,0.53}{\textbf{#1}}}
\newcommand{\OperatorTok}[1]{\textcolor[rgb]{0.81,0.36,0.00}{\textbf{#1}}}
\newcommand{\BuiltInTok}[1]{#1}
\newcommand{\ExtensionTok}[1]{#1}
\newcommand{\PreprocessorTok}[1]{\textcolor[rgb]{0.56,0.35,0.01}{\textit{#1}}}
\newcommand{\AttributeTok}[1]{\textcolor[rgb]{0.77,0.63,0.00}{#1}}
\newcommand{\RegionMarkerTok}[1]{#1}
\newcommand{\InformationTok}[1]{\textcolor[rgb]{0.56,0.35,0.01}{\textbf{\textit{#1}}}}
\newcommand{\WarningTok}[1]{\textcolor[rgb]{0.56,0.35,0.01}{\textbf{\textit{#1}}}}
\newcommand{\AlertTok}[1]{\textcolor[rgb]{0.94,0.16,0.16}{#1}}
\newcommand{\ErrorTok}[1]{\textcolor[rgb]{0.64,0.00,0.00}{\textbf{#1}}}
\newcommand{\NormalTok}[1]{#1}
\usepackage{graphicx,grffile}
\makeatletter
\def\maxwidth{\ifdim\Gin@nat@width>\linewidth\linewidth\else\Gin@nat@width\fi}
\def\maxheight{\ifdim\Gin@nat@height>\textheight\textheight\else\Gin@nat@height\fi}
\makeatother
% Scale images if necessary, so that they will not overflow the page
% margins by default, and it is still possible to overwrite the defaults
% using explicit options in \includegraphics[width, height, ...]{}
\setkeys{Gin}{width=\maxwidth,height=\maxheight,keepaspectratio}
\IfFileExists{parskip.sty}{%
\usepackage{parskip}
}{% else
\setlength{\parindent}{0pt}
\setlength{\parskip}{6pt plus 2pt minus 1pt}
}
\setlength{\emergencystretch}{3em}  % prevent overfull lines
\providecommand{\tightlist}{%
  \setlength{\itemsep}{0pt}\setlength{\parskip}{0pt}}
\setcounter{secnumdepth}{5}
% Redefines (sub)paragraphs to behave more like sections
\ifx\paragraph\undefined\else
\let\oldparagraph\paragraph
\renewcommand{\paragraph}[1]{\oldparagraph{#1}\mbox{}}
\fi
\ifx\subparagraph\undefined\else
\let\oldsubparagraph\subparagraph
\renewcommand{\subparagraph}[1]{\oldsubparagraph{#1}\mbox{}}
\fi

%%% Use protect on footnotes to avoid problems with footnotes in titles
\let\rmarkdownfootnote\footnote%
\def\footnote{\protect\rmarkdownfootnote}

%%% Change title format to be more compact
\usepackage{titling}

% Create subtitle command for use in maketitle
\newcommand{\subtitle}[1]{
  \posttitle{
    \begin{center}\large#1\end{center}
    }
}

\setlength{\droptitle}{-2em}

  \title{Patient-Level Prediction Installation Guide}
    \pretitle{\vspace{\droptitle}\centering\huge}
  \posttitle{\par}
    \author{Jenna Reps, Peter R. Rijnbeek}
    \preauthor{\centering\large\emph}
  \postauthor{\par}
      \predate{\centering\large\emph}
  \postdate{\par}
    \date{2018-09-24}

\usepackage{fancyhdr}
\pagestyle{fancy}
\fancyhead{}
\fancyhead[CO,CE]{Installation Guide}
\fancyfoot[CO,CE]{PatientLevelPrediction Package Version 2.0.5}
\fancyfoot[LE,RO]{\thepage}
\renewcommand{\headrulewidth}{0.4pt}
\renewcommand{\footrulewidth}{0.4pt}

\begin{document}
\maketitle

{
\setcounter{tocdepth}{2}
\tableofcontents
}
\section{Introduction}\label{introduction}

This vignette describes how you need to install the Observational Health
Data Sciencs and Informatics (OHDSI)
\href{http://github.com/OHDSI/PatientLevelPrediction}{\texttt{PatientLevelPrediction}}
package under Windows, Mac, and Linux.

\section{Software Prerequisites}\label{software-prerequisites}

\subsection{Windows Users}\label{windows-users}

Under Windows the OHDSI Patient Level Prediction (PLP) package requires
installing:

\begin{itemize}
\tightlist
\item
  R (\url{https://cran.cnr.berkeley.edu/} ) - (R \textgreater{}= 3.3.0,
  but latest is recommended)
\item
  Rstudio (\url{https://www.rstudio.com/} )
\item
  Java (\url{http://www.java.com} )
\item
  RTools (\url{https://cran.r-project.org/bin/windows/Rtools/})
\item
  Anaconda 3.6 (\url{https://www.anaconda.com/download}) - this will
  require checking your path variable to ensure the correct python is
  used by R - more instructions below. For Python you need to make sure
  it is in the Path: go to my computer -\textgreater{} system properties
  -\textgreater{} advanced system settings Then at the bottom right
  you'll see a button: Environmental Variables, clicking on that will
  enable you to edit the PATH variable to add the Anaconda location. In
  R you need to check the Path is correct: You can access the path
  variable in R using
  \texttt{Sys.getenv(\textquotesingle{}PATH\textquotesingle{})}. This
  should contain the location of your Anaconda or python 3.6.
\item
  If you have Anaconda and want to use PyTorch v0.4
  (\url{https://pytorch.org}) as the backend of deep learning, you can
  directly use command ``conda install pytorch torchvision -c pytorch''
  for Linux. Please refers to commands for installing PyTorch
  (\url{https://pytorch.org}) on other develop environments.
\item
  To add the R keras interface, in Rstudio run:
\end{itemize}

\begin{Shaded}
\begin{Highlighting}[]
\NormalTok{devtools}\OperatorTok{::}\KeywordTok{install_github}\NormalTok{(}\StringTok{"rstudio/keras"}\NormalTok{)}
\KeywordTok{library}\NormalTok{(keras)}
\KeywordTok{install_keras}\NormalTok{()}
\end{Highlighting}
\end{Shaded}

\subsection{Mac/Linux Users}\label{maclinux-users}

Under Mac and Linux the OHDSI Patient Level Prediction (PLP) package
requires installing:

\begin{itemize}
\tightlist
\item
  R (\url{https://cran.cnr.berkeley.edu/} ) - (R \textgreater{}= 3.3.0,
  but latest is recommended)
\item
  Rstudio (\url{https://www.rstudio.com/} )
\item
  Java (\url{http://www.java.com} )
\item
  Xcode command line tools(run in terminal: xcode-select --install)
  {[}MAC USERS ONLY{]}
\item
  Python 3.6 (\url{https://www.python.org/downloads/}) - this will
  require checking your path variable to ensure this version python is
  added - more instructions below
\item
  To add the R keras interface, in Rstudio run:
\end{itemize}

\begin{Shaded}
\begin{Highlighting}[]
\NormalTok{devtools}\OperatorTok{::}\KeywordTok{install_github}\NormalTok{(}\StringTok{"rstudio/keras"}\NormalTok{)}
\KeywordTok{library}\NormalTok{(keras)}
\KeywordTok{install_keras}\NormalTok{()}
\end{Highlighting}
\end{Shaded}

\subsubsection{Setting up Python for Mac/Linux
Users}\label{setting-up-python-for-maclinux-users}

After installing python 3.6 check it is working by typing python3 to
open python in a terminal.

To get the package dependencies, in a terminal run:

\begin{verbatim}
pip3 install --upgrade pip
pip3 install -U NumPy
pip3 install -U SciPy 
pip3 install -U scikit-learn
pip3 install -U torch
pip3 install --upgrade tensorflow 
pip3 install keras
\end{verbatim}

Dependent on your permissions you may need to add a sudo command in
front of the pip3 commands.

Mac and Linux users need edit the bash profile to add python in their
Path by running in the terminal:
\texttt{touch\ \textasciitilde{}/.bash\_profile;\ open\ \textasciitilde{}/.bash\_profile;}
and adding in the location of python 3.6 in the PATH variable. You can
find the location of the python versions by typing this in a terminal:

\begin{verbatim}
type -a python
\end{verbatim}

Furthermore, you need to specify in their R environment that R needs to
use python 3.6 rather than the default python. In a new Rstudio session
run this to open the environment file:

\begin{Shaded}
\begin{Highlighting}[]
\KeywordTok{install.packages}\NormalTok{(‘usethis’)}
\NormalTok{usethis}\OperatorTok{::}\KeywordTok{edit_r_environ}\NormalTok{()}
\end{Highlighting}
\end{Shaded}

In the file that opens add and save: PATH= \{The path containing the
python 3\}

USESPECIALPYTHONVERSION=``python3.6''

You now need to compile PythonInR so it uses python 3.6. In a new R
studio session run:

\begin{Shaded}
\begin{Highlighting}[]
\KeywordTok{Sys.setenv}\NormalTok{(}\StringTok{'USESPECIALPYTHONVERSION'}\NormalTok{=}\StringTok{'python3.6'}\NormalTok{)}
\NormalTok{devtools}\OperatorTok{::}\KeywordTok{install_bitbucket}\NormalTok{(}\StringTok{"Floooo/PythonInR"}\NormalTok{)}
\end{Highlighting}
\end{Shaded}

This should now set the PythonInR package to use your python 3.6. Please
note: if you update the path while R is open, you will need to shutdown
R and reopen before the path is refreshed.

\section{Installing the Package}\label{installing-the-package}

The preferred way to install the package is by using drat, which will
automatically install the latest release and all the latest
dependencies. If the drat code fails or you do not want the official
release you could use devtools to install the bleading edge version of
the package (latest master). Note that the latest master could contain
bugs, please report them to us if you experience problems.

\subsection{Installing PatientLevelPrediction using
drat}\label{installing-patientlevelprediction-using-drat}

To install using drat run:

\begin{Shaded}
\begin{Highlighting}[]
\KeywordTok{install.packages}\NormalTok{(}\StringTok{"drat"}\NormalTok{)}
\NormalTok{drat}\OperatorTok{::}\KeywordTok{addRepo}\NormalTok{(}\StringTok{"OHDSI"}\NormalTok{)}
\KeywordTok{install.packages}\NormalTok{(}\StringTok{"PatientLevelPrediction"}\NormalTok{)}

\NormalTok{## Installing PatientLevelPrediction using devtools}
\NormalTok{To install using devtools run}\OperatorTok{:}
\end{Highlighting}
\end{Shaded}

\begin{Shaded}
\begin{Highlighting}[]
\KeywordTok{install.packages}\NormalTok{(}\StringTok{"devtools"}\NormalTok{)}
\KeywordTok{library}\NormalTok{(}\StringTok{"devtools"}\NormalTok{)}
\KeywordTok{install_github}\NormalTok{(}\StringTok{"ohdsi/SqlRender"}\NormalTok{) }
\KeywordTok{install_github}\NormalTok{(}\StringTok{"ohdsi/DatabaseConnectorJars"}\NormalTok{) }
\KeywordTok{install_github}\NormalTok{(}\StringTok{"ohdsi/DatabaseConnector"}\NormalTok{) }
\KeywordTok{install_github}\NormalTok{(}\StringTok{"ohdsi/FeatureExtraction"}\NormalTok{)}
\KeywordTok{install_github}\NormalTok{(}\StringTok{"ohdsi/OhdsiSharing"}\NormalTok{) }
\KeywordTok{install_github}\NormalTok{(}\StringTok{"ohdsi/OhdsiRTools"}\NormalTok{) }
\KeywordTok{install_github}\NormalTok{(}\StringTok{"ohdsi/BigKnn"}\NormalTok{)  }
\KeywordTok{install_github}\NormalTok{(}\StringTok{"ohdsi/PatientLevelPrediction"}\NormalTok{) }
\end{Highlighting}
\end{Shaded}

\section{Testing installation}\label{testing-installation}

To test whether the package is installed correctly run:

\begin{Shaded}
\begin{Highlighting}[]
\KeywordTok{library}\NormalTok{(DatabaseConnector)}
\NormalTok{connectionDetails <-}\StringTok{ }\KeywordTok{createConnectionDetails}\NormalTok{(}\DataTypeTok{dbms =} \StringTok{'sql_server'}\NormalTok{, }
                                             \DataTypeTok{user =} \StringTok{'username'}\NormalTok{, }
                                             \DataTypeTok{password =} \StringTok{'hidden'}\NormalTok{, }
                                             \DataTypeTok{server =} \StringTok{'your server'}\NormalTok{, }
                                             \DataTypeTok{port =} \StringTok{'your port'}\NormalTok{)}
\NormalTok{PatientLevelPrediction}\OperatorTok{::}\KeywordTok{checkPlpInstallation}\NormalTok{(}\DataTypeTok{connectionDetails =}\NormalTok{ connectionDetails, }
                                             \DataTypeTok{python =}\NormalTok{ T)}
\end{Highlighting}
\end{Shaded}

To test the installation (excluding python) run:

\begin{Shaded}
\begin{Highlighting}[]
\KeywordTok{library}\NormalTok{(DatabaseConnector)}
\NormalTok{connectionDetails <-}\StringTok{ }\KeywordTok{createConnectionDetails}\NormalTok{(}\DataTypeTok{dbms =} \StringTok{'sql_server'}\NormalTok{, }
                                           \DataTypeTok{user =} \StringTok{'username'}\NormalTok{, }
                                           \DataTypeTok{password =} \StringTok{'hidden'}\NormalTok{, }
                                           \DataTypeTok{server =} \StringTok{'your server'}\NormalTok{, }
                                           \DataTypeTok{port =} \StringTok{'your port'}\NormalTok{)}
\NormalTok{PatientLevelPrediction}\OperatorTok{::}\KeywordTok{checkPlpInstallation}\NormalTok{(}\DataTypeTok{connectionDetails =}\NormalTok{ connectionDetails, }
                                             \DataTypeTok{python =}\NormalTok{ F)}
\end{Highlighting}
\end{Shaded}

The check can take a while to run since it will build the following
models in sequence on simulated \url{data:Logistic} Regression,
RandomForest, MLP, AdaBoost, Decision Tree, Naive Bayes, KNN, Gradient
Boosting. Moreover, it will test the database connection.

\section{Installation issues}\label{installation-issues}

Installation issues need to be posted in our issue tracker:
\url{http://github.com/OHDSI/PatientLevelPrediction/issues}

The list below provides solutions for some common issues:

\begin{enumerate}
\def\labelenumi{\arabic{enumi}.}
\item
  If you have an error when trying to install a package in R saying
  \textbf{`Dependancy X not available \ldots{}'} then this can sometimes
  be fixed by running
  \texttt{install.packages(\textquotesingle{}X\textquotesingle{})} and
  then once that completes trying to reinstall the package that had the
  error.
\item
  I have found that using the github devtools to install packages can be
  impacted if you have \textbf{multiple R sessions} open as one session
  with a library open can causethe library to be locked and this can
  prevent an install of a package that depends on that library.
\end{enumerate}

\section{Acknowledgments}\label{acknowledgments}

Considerable work has been dedicated to provide the
\texttt{PatientLevelPrediction} package.

\begin{Shaded}
\begin{Highlighting}[]
\KeywordTok{citation}\NormalTok{(}\StringTok{"PatientLevelPrediction"}\NormalTok{)}
\end{Highlighting}
\end{Shaded}

\begin{verbatim}
## 
##   Jenna Reps, Martijn J. Schuemie, Marc A. Suchard, Patrick B.
##   Ryan and Peter R. Rijnbeek (2018). PatientLevelPrediction:
##   Package for patient level prediction using data in the OMOP
##   Common Data Model. R package version 2.0.5.
## 
## A BibTeX entry for LaTeX users is
## 
##   @Manual{,
##     title = {PatientLevelPrediction: Package for patient level prediction using data in the OMOP Common Data
## Model},
##     author = {Jenna Reps and Martijn J. Schuemie and Marc A. Suchard and Patrick B. Ryan and Peter R. Rijnbeek},
##     year = {2018},
##     note = {R package version 2.0.5},
##   }
\end{verbatim}

\textbf{Please reference this paper if you use the PLP Package in your
work:}

\href{http://dx.doi.org/10.1093/jamia/ocy032}{Reps JM, Schuemie MJ,
Suchard MA, Ryan PB, Rijnbeek PR. Design and implementation of a
standardized framework to generate and evaluate patient-level prediction
models using observational healthcare data. J Am Med Inform Assoc.
2018;25(8):969-975.}

This work is supported in part through the National Science Foundation
grant IIS 1251151.


\end{document}
